\section{Conclusion}
  \label{sec:conclusion}

  We have examined whether two long-standing observational anomalies,
  namely flat galaxy rotation curves and the Hubble constant tension,
  may originate from a common effective mechanism operating in low-density environments.

  Using a minimal phenomenological framework characterized by a single saturation
  scale, we have shown that baryonic matter alone can reproduce the main features
  of observed galaxy rotation curves across a wide range of systems, without
  halo-by-halo tuning or the introduction of dark matter particles within this
  effective description.
  This conclusion is supported by direct comparisons with
  representative rotation curves from the SPARC sample.

  The same effective mechanism extends naturally to cosmological low-density regions.
  Cosmic voids, which probe the deepest saturation regime, emerge as key laboratories for
  this framework.
  In these environments, the locally inferred expansion rate is predicted to deviate systematically
  from the global value inferred from early-universe observables,
  while the background cosmological evolution remains unchanged.

  Crucially, the framework does not modify early-time physics,
  does not alter the global expansion history, and does not claim universal validity across all regimes.
  Quantities such as gravitational acceleration and the Hubble parameter
  are interpreted as effective inference rates, not as fundamental dynamical fields.

  A distinctive strength of the approach lies in its cross-sector falsifiability.
  Because the same saturation scale governs both galactic dynamics and
  environment-dependent expansion inference, observational constraints obtained
  in one sector directly restrict the allowed phenomenology in the other.
  In particular, measurements of ultra-diffuse galaxy kinematics,
  which probe the deepest saturation regime at galactic scales,
  place quantitative bounds on the magnitude of any environment-dependent
  offset in the inferred Hubble constant.
  Conversely, the absence of such an offset in cosmological observations
  would falsify the interpretation of galactic low-acceleration behavior
  as a saturation effect.

  The strength of the framework thus lies not in fitting individual anomalies
  in isolation, but in tying together galactic and cosmological phenomenology
  through a single effective scale and a shared environmental dependence.
  This coherence sharply limits the available parameter space
  and prevents independent adjustment of the two sectors.

  We have identified several observational tests capable of confirming or excluding the framework.
  These include correlations between inferred values of $H_0$ and large-scale environment,
  redshift-dependent suppression of the local offset, void-specific redshift drift,
  weak lensing signatures, and the dynamics of ultra-diffuse galaxies.

  Taken together, these results suggest that galaxy rotation curves and the
  Hubble tension may reflect a shared low-density phenomenology rather than
  independent failures of the standard cosmological model.
  Whether this effective description captures a genuine aspect of gravitational
  inference or represents an intermediate phenomenological layer will ultimately be
  decided by targeted observational tests in the coming years.
