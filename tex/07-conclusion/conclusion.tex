\section{Conclusion}
  \label{sec:conclusion}

  We have investigated whether two persistent observational anomalies---flat
  galaxy rotation curves and the Hubble constant tension---may originate from a
  common effective mechanism operating in low-density environments.

  Using a minimal phenomenological framework characterized by a single saturation
  scale, we have shown that baryonic matter alone can account for the main features
  of observed galaxy rotation curves across a wide range of systems, without
  halo-by-halo tuning or the introduction of dark matter particles in this
  effective description.
  This conclusion is supported by explicit numerical comparisons with
  representative rotation curves drawn from the SPARC sample.
  The same mechanism naturally extends to cosmological low-density regions, where
  it predicts environment-dependent deviations in the locally inferred expansion
  rate.


  Cosmic voids emerge as key laboratories for testing this framework.
  By probing the deeply saturated regime, they provide a structural explanation
  for why late-time local measurements of the Hubble constant may systematically
  exceed the global value inferred from early-universe observables, while leaving
  the background cosmological evolution unchanged.

  The model is intentionally limited in scope.
  It does not modify early-time physics, does not alter the global expansion
  history, and does not claim universal validity across all regimes.
  Its strength lies instead in its parsimony and falsifiability.
  All predictions follow from a single effective scale already constrained by
  galactic dynamics.

  We have identified several observational tests capable of confirming or
  excluding this framework, including correlations between $H_0$ measurements and
  large-scale environment, redshift-dependent suppression of the local offset,
  void-specific redshift drift, weak lensing signatures, and the dynamics of
  ultra-diffuse galaxies.

  Taken together, these results suggest that galaxy rotation curves and the
  Hubble tension may reflect a shared low-density phenomenology rather than
  independent failures of the standard cosmological model.
  Whether this effective description captures a genuine aspect of gravitational
  physics or represents an intermediate phenomenological layer will ultimately be
  decided by observational tests in the coming years.
