\begin{abstract}
  We investigate whether two long-standing observational anomalies---flat galaxy
  rotation curves and the Hubble constant tension---may originate from a common
  effective mechanism operating in low-density environments.

  We introduce a minimal phenomenological framework in which departures from
  Newtonian expectations arise from a saturation of the effective gravitational
  response, without invoking dark matter particles or altering early-universe
  cosmology.
  The framework reduces to standard Newtonian behavior in high-density regimes
  and introduces a smooth, bounded effective response in diffuse environments.

  Applied to galactic dynamics, this approach reproduces the main features of
  observed rotation curves across a wide range of galaxy types using baryonic
  matter alone, with stable parameter values and no halo-by-halo tuning.
  Explicit numerical comparisons with representative galaxies from the SPARC
  sample show that the effective saturation scale naturally correlates with
  observed surface-density-dependent trends.

  At cosmological scales, the same effective mechanism predicts
  environment-dependent deviations in the locally inferred expansion rate.
  Cosmic voids emerge as maximal probes of the saturated regime, leading to a
  systematic offset between local and global measurements of the Hubble constant,
  while preserving the background cosmological evolution.
  This provides a structural interpretation of the Hubble tension without
  introducing new dark components or modifying early-time physics.

  We examine robustness, degeneracies with standard astrophysical effects, and
  limitations of the effective description, and we outline distinctive
  observational signatures, including environment-dependent redshift drift and
  weak lensing effects.
  The results suggest that galaxy rotation curves and the Hubble tension may
  reflect a shared low-density phenomenology, testable with current and
  forthcoming observations.
\end{abstract}
