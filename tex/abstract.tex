\begin{abstract}
  We investigate whether two long-standing observational anomalies—flat galaxy rotation curves and the Hubble constant
  tension—may originate from a single effective modification of gravitational dynamics operating in low-density regimes.

  We introduce a minimal phenomenological model in which the effective gravitational response saturates below a
  characteristic density or acceleration scale, without invoking dark matter particles or altering early-universe
  cosmology.
  The model reduces to Newtonian gravity in high-density environments and introduces a smooth, bounded modification in
  diffuse regions.

  Applied to galactic dynamics, this framework reproduces the main features of observed rotation curves across a wide
  range of galaxy types using baryonic matter alone, with stable parameter values and no halo-by-halo tuning.
  Numerical fits to the SPARC sample demonstrate that the effective saturation scale correlates naturally with observed
  surface density trends.

  At cosmological scales, the same mechanism predicts environment-dependent deviations in the locally inferred expansion
  rate.
  Cosmic voids emerge as maximal probes of the saturated regime, leading to a systematic offset between local and global
  measurements of the Hubble constant.
  This provides a structural explanation for the Hubble tension without introducing new dark components or modifying
  early-time physics.

  We discuss degeneracies, limitations, and robustness against baryonic uncertainties, and we outline distinctive
  observational signatures, including void-dependent redshift drift and lensing effects.
  The results suggest that galaxy rotation curves and the Hubble tension may reflect a common low-density phenomenology,
  testable with current and forthcoming observations.
\end{abstract}
