\section{Effective Model}
  \label{sec:effective_model}

  This section introduces a minimal effective framework designed to capture
  gravitational phenomenology in low-density environments.
  The model is constructed to address galaxy rotation curves and the Hubble
  tension within a single unified description, without modifying early-universe
  physics or invoking dark matter particles in this effective regime.

  The emphasis is deliberately phenomenological.
  Only the minimal assumptions required to reproduce the observed anomalies are
  introduced.
  The model is not presented as a fundamental theory of gravity, but as an
  effective description valid in specific environmental regimes.

  The specific form of Eq.~\eqref{eq:modified_poisson} is adopted here as a
  minimal effective parametrization of a bounded-response regime.
  A dynamical motivation for such saturation effects is developed in a companion
  work, while the present paper restricts attention to late-time observational
  tests~\cite{Beau2026c}.

  \subsection{Minimal Assumptions}
    \label{subsec:minimal_assumptions}

    We assume that gravitational dynamics at galactic and sub-cosmological scales
    can be described by an effective potential $\Phi_{\mathrm{eff}}$ sourced by
    the observed baryonic mass distribution.
    The model satisfies the following minimal requirements.

    First, the effective dynamics must reduce to standard Newtonian gravity in
    high-density environments.
    This ensures consistency with Solar System tests and with the inner regions of
    high-surface-density galaxies.

    Second, deviations from Newtonian behavior are allowed only in diffuse regimes,
    characterized by low baryonic density or low gravitational acceleration.
    The transition between regimes must be smooth and free of singular behavior.

    Third, the effective modification must be governed by at most one additional
    scale parameter, denoted $a_\star$, which controls the onset of saturation.
    No galaxy-dependent or environment-specific tuning is introduced.

    Fourth, the model must preserve locality and isotropy at the effective level.
    No non-local interactions or preferred directions are assumed.

    Finally, the model must admit a well-defined cosmological interpretation at
    late times, without altering early-universe observables such as primordial
    nucleosynthesis or recombination physics.

  \subsection{Effective Saturating Potential}
    \label{subsec:effective_potential}

    We define the effective gravitational potential $\Phi_{\mathrm{eff}}$ through
    a modified Poisson equation of the form
    \begin{equation}
      \nabla \cdot \left[ \mu\!\left( \frac{|\nabla \Phi_{\mathrm{eff}}|}{a_\star} \right)
                     \nabla \Phi_{\mathrm{eff}} \right]
      =
      4 \pi G \rho_{\mathrm{b}} ,
      \label{eq:modified_poisson}
    \end{equation}
    where $\rho_{\mathrm{b}}$ denotes the baryonic mass density.
    The dimensionless function $\mu(x)$ encodes the effective saturation of the
    gravitational response.

    The function $\mu(x)$ satisfies the limiting behaviors
    \begin{equation}
      \mu(x) \rightarrow 1 \quad \text{for} \quad x \gg 1 ,
    \end{equation}
    and
    \begin{equation}
      \mu(x) \rightarrow x \quad \text{for} \quad x \ll 1 .
    \end{equation}

    In the high-acceleration regime $|\nabla \Phi_{\mathrm{eff}}| \gg a_\star$,
    Eq.~\eqref{eq:modified_poisson} reduces to the standard Poisson equation,
    recovering Newtonian gravity.
    In the low-acceleration regime, the effective response saturates, leading to an
    enhanced gravitational influence at large radii.

    For spherically symmetric systems, the effective radial acceleration
    $g_{\mathrm{eff}}(r)$ satisfies
    \begin{equation}
      \mu\!\left( \frac{g_{\mathrm{eff}}}{a_\star} \right) g_{\mathrm{eff}}
      =
      g_{\mathrm{N}} ,
      \label{eq:effective_acceleration}
    \end{equation}
    where $g_{\mathrm{N}}$ is the Newtonian acceleration sourced by baryons alone.

    Equation~\eqref{eq:effective_acceleration} yields asymptotically flat rotation
    curves for isolated galaxies with finite baryonic mass.
    The transition scale $a_\star$ controls both the onset of flattening and the
    amplitude of the asymptotic velocity.

    Importantly, $a_\star$ is treated as a universal parameter in this work.
    No halo-by-halo fitting or galaxy-dependent adjustment is introduced.
    All diversity in rotation curve shapes arises from the observed baryonic
    distributions.

  \subsection{Environmental Dependence and Low-Density Regimes}
    \label{subsec:environment}

    The effective modification introduced above is intrinsically sensitive to the
    local gravitational environment.
    Regions of low baryonic density or weak gravitational gradients probe the
    saturated regime most strongly.

    This environmental dependence plays a central role at cosmological scales.
    Cosmic voids, characterized by extremely low matter density, naturally sample
    the deep saturation regime of the effective dynamics.
    As a result, local kinematic measurements performed within void-dominated
    regions may exhibit systematic deviations from globally inferred expansion
    rates.

    In this framework, galaxy rotation curves and the Hubble tension arise as two
    manifestations of the same low-density phenomenology.
    Galaxies probe saturation radially, while cosmic voids probe it volumetrically.

    The following sections test this effective model against galactic rotation
    curve data and explore its implications for local measurements of the Hubble
    constant.
