\section{Effective Model}
  \label{sec:effective_model}

  This section introduces a minimal effective framework designed to capture
  late-time gravitational phenomenology in low-density environments.
  The model is constructed to address galaxy rotation curves and the Hubble
  tension within a single unified description,
  without modifying early-universe physics or invoking dark matter particles
  in this effective regime.

  The emphasis is deliberately phenomenological.
  Only the minimal assumptions required to reproduce the observed anomalies are
  introduced.
  The model is not presented as a fundamental theory of gravity,
  but as an effective description valid in specific environmental regimes.

  The guiding interpretation adopted here is that late-time gravitational
  observables characterize effective transition rates between admissible
  macroscopic configurations of the observable Universe.
  In this view, quantities such as gravitational acceleration or the Hubble
  parameter do not represent fundamental dynamical fields.
  They encode how rapidly the effective state of the Universe can be updated
  under a finite resolution of the underlying descriptive structure.

  Saturation is interpreted as a limitation of this effective resolution.
  In low-density environments, the number of independent relations contributing
  to the gravitational response becomes bounded,
  so that further decreases in density no longer translate into proportionally
  stronger inferred dynamics.
  As a result, the effective response approaches a plateau rather than vanishing
  asymptotically.
  The dynamical motivation for bounded-response regimes of this type is developed
  in a companion theoretical work~\cite{Beau2026c}.

  \subsection{Minimal Assumptions and Domain of Validity}
    \label{subsec:minimal_assumptions}

    We assume that gravitational phenomenology at galactic and sub-cosmological
    scales can be described by an effective potential $\Phi_{\mathrm{eff}}$,
    used here as a convenient parametrization of kinematic inference rather than
    as a fundamental dynamical field.
    The framework satisfies the following requirements.

    (i) The effective dynamics must reduce to standard Newtonian gravity in
    high-density environments,
    ensuring consistency with Solar System tests.
    (ii) Deviations are allowed only in diffuse regimes,
    characterized by low baryonic density or weak gravitational gradients.
    (iii) The transition is governed by at most one additional universal scale
    parameter $a_\star$,
    with no environment-specific tuning.
    (iv) Locality and isotropy are preserved at the effective level.

    Mathematically, the framework shares the Lagrangian structure of the
    quasi-linear MOND theories proposed by Bekenstein and Milgrom~\cite{Bekenstein1984}.
    However, whereas MOND introduces this modification as a fundamental alteration
    of the gravitational law,
    the present approach interprets it as a saturated effective response.
    The modification reflects a finite capacity of the effective description
    to resolve increasingly dilute relational structure~\cite{Beau2026b}.

    This conceptual shift is essential.
    It allows the phenomenological success of MOND at galactic scales to be
    reinterpreted as a regime of descriptive saturation,
    rather than as a modification of gravity.
    The same mechanism then applies naturally to volumetric density dilution
    in cosmological voids.

    The present framework is therefore not intended as a replacement for MOND
    phenomenology.
    It provides an interpretation of its low-acceleration successes
    as an emergent effective limit within a broader descriptive structure.

  \subsection{Effective Saturating Potential}
    \label{subsec:effective_potential}

    We define the effective gravitational potential $\Phi_{\mathrm{eff}}$ through a
    modified Poisson equation of the form
    \begin{equation}
      \nabla \cdot \left[
                     \mu\!\left( \frac{|\nabla \Phi_{\mathrm{eff}}|}{a_\star} \right)
                     \nabla \Phi_{\mathrm{eff}}
      \right]
      =
      4 \pi G \rho_{\mathrm{b}} ,
      \label{eq:modified_poisson}
    \end{equation}
    where $\rho_{\mathrm{b}}$ denotes the baryonic mass density
    and $\mu(x)$ encodes the saturation of the effective response.

    For clarity, we work primarily with the $\mu$-formulation in
    Eq.~\eqref{eq:modified_poisson}.
    An equivalent algebraic form is sometimes written using an interpolation
    function $\nu(y)$ defined by $g_{\mathrm{eff}} = \nu(y) g_{\mathrm{N}}$
    with $y = g_{\mathrm{N}}/a_\star$.
    In that notation, $\nu(y) \rightarrow 1$ for $y \gg 1$,
    ensuring an exactly Newtonian limit.

    In the present interpretation, $\mu(x)$ does not represent a fundamental
    field-dependent coupling.
    It parametrizes the progressive saturation of the effective response
    as the underlying transition structure approaches its maximal resolvable rate.
    The specific functional form adopted here provides a minimal interpolation
    between regimes and is not intended as a fundamental law.

    The function $\mu(x)$ satisfies the limiting behaviors
    \begin{equation}
      \mu(x) \rightarrow 1 \quad \text{for} \quad x \gg 1 ,
    \end{equation}
    and
    \begin{equation}
      \mu(x) \rightarrow x \quad \text{for} \quad x \ll 1 .
    \end{equation}

    We assume $\mu(x)$ to be positive, monotone increasing, and at least $C^{1}$,
    so that $\Phi_{\mathrm{eff}}$ and its first derivatives remain continuous
    across the transition.
    This regularity avoids spurious discontinuities in the effective force
    and ensures stable orbital dynamics.

    In the high-acceleration regime,
    Eq.~\eqref{eq:modified_poisson} reduces to the standard Poisson equation.
    In the low-acceleration or diffuse regime,
    the effective response saturates,
    leading to a finite asymptotic acceleration.

    For spherically symmetric systems,
    the effective radial acceleration $g_{\mathrm{eff}}(r)$ satisfies
    \begin{equation}
      \mu\!\left( \frac{g_{\mathrm{eff}}}{a_\star} \right)
      g_{\mathrm{eff}}
      =
      g_{\mathrm{N}} ,
      \label{eq:effective_acceleration}
    \end{equation}
    where $g_{\mathrm{N}}$ is the Newtonian acceleration sourced by baryons alone.

    Equation~\eqref{eq:effective_acceleration} yields asymptotically flat rotation
    curves for isolated galaxies with finite baryonic mass.
    The transition scale $a_\star$ controls both the onset of flattening
    and the amplitude of the asymptotic velocity.

    Importantly, $a_\star$ is treated as a universal parameter.
    No halo-by-halo fitting or galaxy-dependent adjustment is introduced.
    All diversity in rotation curve shapes arises from the observed baryonic
    distributions.
    In this work, $a_\star$ is interpreted as a minimal resolvable acceleration
    within the effective description,
    reflecting an intrinsic bound on the transition rate
    of the geometric response.

  \subsection{Environmental Dependence and Low-Density Regimes}
    \label{subsec:environment}

    The effective modification introduced above is intrinsically sensitive to the
    gravitational environment.
    Regions of low baryonic density or weak gravitational gradients
    probe the saturated regime most strongly.

    This environmental dependence becomes critical at cosmological scales.
    Cosmic voids, characterized by extreme matter dilution,
    naturally sample the deep saturation regime.
    As a result, local kinematic measurements performed in void-dominated regions
    may exhibit systematic offsets relative to globally inferred expansion rates.

    Within this framework, the Hubble tension is not interpreted
    as a failure of early-universe cosmology.
    It arises as a late-time inference bias,
    reflecting the use of a single spacetime-based parametrization
    across environments characterized by different effective resolutions.

    Galaxy rotation curves and the Hubble tension therefore emerge
    as two manifestations of the same low-density phenomenology.
    Galaxies probe saturation radially through declining surface density,
    while cosmic voids probe it volumetrically through large-scale dilution.
    The identification of a single scale $a_\star$
    accounting for both effects constitutes the central organizing principle of this effective description.

    The following sections test this effective framework
    against galactic rotation curve data
    and explore its implications for local measurements
    of the Hubble constant.
