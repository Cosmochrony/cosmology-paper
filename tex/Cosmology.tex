\documentclass{aa}

\usepackage{graphicx}
\usepackage{txfonts}
\usepackage{natbib}
\usepackage{hyperref}

\begin{document}

  \title{A unified effective mechanism for galaxy rotation curves and the Hubble tension}

  \author{J\'er\^ome Beau}
  \institute{Independent Researcher, France\\
  \email{jerome.beau@cosmochrony.org}
  }

  \date{Received XXX; accepted YYY}

  \abstract{
    We investigate whether two long-standing observational anomalies---flat galaxy
    rotation curves and the Hubble constant tension---may originate from a single
    effective modification of gravitational response operating in low-density
    regimes.

    We introduce a minimal phenomenological model in which the effective
    gravitational response saturates below a characteristic acceleration scale,
    without invoking dark matter particles or altering early-universe cosmology.
    The model reduces to Newtonian gravity in high-density environments and
    introduces a smooth, bounded modification in diffuse regions.

    Applied to galactic dynamics, this framework reproduces the main features of
    observed rotation curves across a wide range of galaxy types using baryonic
    matter alone, with stable parameter values and no halo-by-halo tuning.
    Explicit numerical fits to representative galaxies from the SPARC sample
    demonstrate that the effective saturation scale correlates naturally with
    observed surface-density trends.

    At cosmological scales, the same mechanism predicts environment-dependent
    deviations in the locally inferred expansion rate.
    Cosmic voids emerge as maximal probes of the saturated regime, leading to a
    systematic offset between local and global measurements of the Hubble constant.
    This provides a structural explanation for the Hubble tension without
    introducing new dark components or modifying early-time physics.

    We discuss degeneracies, limitations, and robustness against baryonic
    uncertainties, and we outline distinctive observational signatures, including
    void-dependent redshift drift and lensing effects.
    The results suggest that galaxy rotation curves and the Hubble tension may
    reflect a common low-density phenomenology, testable with current and
    forthcoming observations.
  }

  \maketitle

  \section{Introduction}
    \label{sec:introduction}

    Two observational puzzles continue to motivate scrutiny of late-time
    gravitational phenomenology.

    The first is the ubiquity of flat galaxy rotation curves.
    Across a wide range of disk galaxies, circular velocities remain nearly
    constant at large radii, in apparent tension with the expectation from
    Newtonian dynamics applied to the observed baryonic mass distribution.
    This behavior has been firmly established by large and homogeneous datasets,
    most notably the SPARC compilation, which provides resolved rotation curves
    together with self-consistent baryonic mass models for disk galaxies~\cite{Lelli2016SPARC}.
    Within the standard cosmological framework, flat rotation curves are typically
    accounted for by extended dark matter halos.
    While this approach is empirically successful, it introduces substantial model
    freedom at the level of individual galaxies and leaves open the question of why
    tight empirical regularities, such as surface-density-dependent trends and
    acceleration-based scalings, emerge so prominently in the data~\cite{McGaugh2016RAR}.
    Historically, these regularities led to the development of the Modified Newtonian
    Dynamics (MOND) paradigm~\cite{Milgrom1983, Bekenstein1984}, which remains
    the primary phenomenological benchmark for such anomalies.

    The second puzzle is the Hubble constant tension.
    Late-time local determinations of $H_0$ based on distance ladder techniques and
    standard candles remain in significant disagreement with early-time inferences
    derived from the homogeneous high-redshift Universe as constrained by cosmic
    microwave background observations~\cite{Riess2022SH0ES,Planck2020}.
    This discrepancy has persisted across multiple datasets and analysis methods,
    prompting proposals ranging from unidentified systematics to extensions of the
    standard cosmological model.
    Comprehensive reviews indicate that no consensus explanation has yet emerged~\cite{Verde2019Review}.
    A notable feature of the tension is its strong association with late-time
    inference in a structured and inhomogeneous Universe.

    Although galaxy rotation curves and the Hubble tension are usually discussed
    separately, both phenomena probe gravitational and kinematic inference in
    low-density regimes.
    Galaxy outskirts and ultra-diffuse systems explore weak-acceleration regions,
    while cosmic voids represent the most extreme low-density environments on
    cosmological scales.
    This motivates the central question addressed in this work:
    Can a single effective mechanism, active primarily in diffuse environments,
    account for both classes of anomalies within a unified phenomenological
    description?

    In this paper, we investigate a minimal effective framework in which departures
    from Newtonian dynamics arise in low-density environments.
    In such regimes, the effective gravitational response does not continue to grow
    indefinitely as density decreases, but instead approaches a plateau controlled
    by an intrinsic saturation scale.
    While the mathematical structure of this saturation in the galactic limit
    shares similarities with the a-field theories of the Bekenstein-Milgrom
    formulation~\cite{Bekenstein1984}, our approach is conceptually distinct.
    Rather than an ad hoc modification of the force law, this behavior
    reflects a limitation in the effective description applicable to
    diffuse regimes, where dynamical inference becomes progressively
    insensitive to further reductions in density.

    This phenomenological study is motivated by a broader theoretical context in
    which bounded-response regimes (of the Born-Infeld type) arise naturally in
    effective descriptions of gravitational dynamics~\cite{Beau2026c}.
    Related structural and projection-based considerations are discussed in
    companion works focusing on the informational foundations of the geometric
    support~\cite{Beau2026a,Beau2026b}.
    Here we deliberately restrict attention to late-time observational
    consequences and do not rely on the full underlying framework.

    The model is constructed to reduce to standard Newtonian behavior in
    high-density environments and to introduce a smooth bounded modification in
    diffuse regimes.
    It is treated explicitly as an effective description of late-time
    phenomenology.
    No modification of early-universe physics is assumed, and the model is not
    presented as a fundamental theory of gravity.

    The analysis proceeds in two stages.
    First, we test the model against galaxy rotation curve data, focusing on the
    SPARC sample, and evaluate whether a single universal saturation scale can
    reproduce the observed diversity of rotation curve shapes using baryonic matter
    alone, without halo-by-halo tuning.
    Second, we explore the cosmological implications of the same mechanism in
    low-density environments, emphasizing cosmic voids as maximal probes and
    deriving testable signatures for environment-dependent expansion inference.

    A key aim of this work is falsifiability.
    Beyond fitting existing data, we identify observational discriminants that can
    confirm or exclude the framework.
    These include correlations between locally inferred values of $H_0$ and
    large-scale environment, redshift-dependent suppression of the local offset,
    void-specific redshift drift, and weak lensing signatures.

    The structure of the paper is as follows.
    Section~\ref{sec:effective_model} introduces the effective model and its minimal
    assumptions.
    Section~\ref{sec:rotation_curves} presents rotation curve tests and diagnostics.
    Section~\ref{sec:voids_hubble} discusses the implications for cosmic voids and
    the Hubble tension.
    Section~\ref{sec:robustness} examines robustness, degeneracies, and limits.
    Section~\ref{sec:predictions} summarizes predictions and observational tests.
    Section~\ref{sec:conclusion} concludes.

  \section{Effective Model}
    \label{sec:effective_model}

    This section introduces a minimal effective framework designed to capture
    gravitational phenomenology in low-density environments.
    The model is constructed to address galaxy rotation curves and the Hubble
    tension within a single unified description, without modifying early-universe
    physics or invoking dark matter particles in this effective regime.

    The emphasis is deliberately phenomenological.
    Only the minimal assumptions required to reproduce the observed anomalies are
    introduced.
    The model is not presented as a fundamental theory of gravity, but as an
    effective description valid in specific environmental regimes.

    In the present framework, saturation is interpreted as a limitation in the
    resolution of structural information available to the effective description.
    In low-density environments, the number of independent relations contributing
    to the gravitational response becomes bounded, so that additional decreases in
    density no longer translate into proportionally stronger dynamical effects.
    As a result, the effective response approaches a plateau rather than vanishing
    asymptotically.
    The dynamical motivation for such bounded-response regimes is developed in a
    companion theoretical work~\cite{Beau2026c}.

    \subsection{Minimal Assumptions and Domain of Validity}
      \label{subsec:minimal_assumptions}

      We assume that gravitational dynamics at galactic and sub-cosmological scales
      can be described by an effective potential $\Phi_{\mathrm{eff}}$ sourced by the
      observed baryonic mass distribution.
      The model satisfies the following requirements:

      (i) The effective dynamics must reduce to standard Newtonian gravity in
      high-density environments, ensuring consistency with Solar System tests.
      (ii) Deviations are allowed only in diffuse regimes, characterized by low
      baryonic density or weak gravitational gradients.
      (iii) The transition is governed by at most one additional universal
      scale parameter $a_\star$, with no environment-specific tuning.
      (iv) The model must preserve locality and isotropy at the effective level.

      Mathematically, this framework shares the Lagrangian structure of the
      quasi-linear MOND theories (AQUAL) proposed by Bekenstein and
      Milgrom~\cite{Bekenstein1984}.
      However, whereas traditional MOND introduces this modification as a fundamental
      alteration of the relation between the gravitational field and its Newtonian
      counterpart, our framework treats the modification as a \textit{saturated response}
      dictated by the intrinsic flux limit of the geometric support~\cite{Beau2026b}.
      This conceptual shift is crucial as it naturally extends the phenomenological
      successes of galactic MOND to the volumetric density dilution of
      cosmological voids.
      ~Notably, the same universal constant $a_\star$ is applied here to both
      localized galactic kinematics and large-scale expansion inference, providing
      a testable unification between the Radial Acceleration Relation and the
      Hubble tension.

      We emphasize that the present framework is not intended as a replacement for MOND
      phenomenology, but as an interpretation of its low-acceleration successes as an
      emergent effective limit within a broader descriptive structure.

    \subsection{Effective Saturating Potential}
      \label{subsec:effective_potential}

      We define the effective gravitational potential $\Phi_{\mathrm{eff}}$ through a
      modified Poisson equation of the form:
      \begin{equation}
        \nabla \cdot \left[
                       \mu\!\left( \frac{|\nabla \Phi_{\mathrm{eff}}|}{a_\star} \right)
                       \nabla \Phi_{\mathrm{eff}}
        \right]
        =
        4 \pi G \rho_{\mathrm{b}} ,
        \label{eq:modified_poisson}
      \end{equation}
      where $\rho_{\mathrm{b}}$ denotes the baryonic mass density and $\mu(x)$ is the
      dimensionless function encoding the saturation of the gravitational response.

      For clarity, we work primarily with the $\mu$-formulation in
      Eq.~\eqref{eq:modified_poisson}.
      An equivalent algebraic form is sometimes written using an interpolation
      function $\nu(y)$ defined by $g_{\mathrm{eff}} = \nu(y) g_{\mathrm{N}}$ with
      $y = g_{\mathrm{N}}/a_\star$.
      In that notation, $\nu(y) \rightarrow 1$ for $y \gg 1$, ensuring an exactly
      Newtonian limit.

      In this formulation, $\mu(x)$ parametrizes the progressive saturation of the
      effective response as the underlying structural flux approaches its maximal
      resolvable value.
      The specific form adopted here provides a minimal interpolation between regimes
      and is not intended as a fundamental law.

      The function $\mu(x)$ satisfies the limiting behaviors
      \begin{equation}
        \mu(x) \rightarrow 1 \quad \text{for} \quad x \gg 1 ,
      \end{equation}
      and
      \begin{equation}
        \mu(x) \rightarrow x \quad \text{for} \quad x \ll 1 .
      \end{equation}

      We assume $\mu(x)$ to be positive, monotone increasing, and at least $C^{1}$,
      so that $\Phi_{\mathrm{eff}}$ and its first derivatives remain continuous
      across the transition.
      This regularity avoids spurious discontinuities in the effective force and
      ensures stable orbital dynamics in the transition region.

      In the high-acceleration regime $|\nabla \Phi_{\mathrm{eff}}| \gg a_\star$,
      Eq.~\eqref{eq:modified_poisson} reduces to the standard Poisson equation,
      recovering Newtonian gravity.
      In the low-acceleration or diffuse regime, the effective response saturates,
      leading to a finite asymptotic force.

      For spherically symmetric systems, the effective radial acceleration
      $g_{\mathrm{eff}}(r)$ satisfies:
      \begin{equation}
        \mu\!\left( \frac{g_{\mathrm{eff}}}{a_\star} \right)
        g_{\mathrm{eff}}
        =
        g_{\mathrm{N}} ,
        \label{eq:effective_acceleration}
      \end{equation}
      where $g_{\mathrm{N}}$ is the Newtonian acceleration sourced by baryons alone.

      Equation~\eqref{eq:effective_acceleration} yields asymptotically flat rotation
      curves for isolated galaxies with finite baryonic mass.
      The transition scale $a_\star$ controls both the onset of flattening and the
      amplitude of the asymptotic velocity.

      Importantly, $a_\star$ is treated as a universal parameter in this work.
      No halo-by-halo fitting or galaxy-dependent adjustment is introduced.
      All diversity in rotation curve shapes arises from the observed baryonic
      distributions.
      In this paper $a_\star$ is treated as an effective acceleration threshold.
      Its universality is interpreted as reflecting an intrinsic bound on the
      maximal resolvable structural flux of the effective geometric response.
      ~Consequently, any environmental volume whose gravitational gradient
      falls below $a_\star$ is expected to exhibit an effective dynamical offset
      governed by this same fundamental scale.

    \subsection{Environmental Dependence and Low-Density Regimes}
      \label{subsec:environment}

      The effective modification introduced above is intrinsically sensitive to the
      local gravitational environment.
      Regions of low baryonic density or weak gravitational gradients probe the
      saturated regime most strongly.

      This environmental dependence plays a central role at cosmological scales.
      Cosmic voids, characterized by extremely low matter density, naturally sample
      the deep saturation regime of the effective dynamics.
      As a result, local kinematic measurements performed within void-dominated
      regions may exhibit systematic deviations from globally inferred expansion
      rates.
      ~The Hubble tension is thus interpreted here not as a failure of early-universe
      cosmology, but as a possible local expansion bias emerging from the saturation
      of the gravitational response in underdense late-time environments.

      In this framework, galaxy rotation curves and the Hubble tension arise as two
      manifestations of the same low-density phenomenology.
      Galaxies probe saturation radially through declining surface density, while
      cosmic voids probe it volumetrically through large-scale dilution.
      ~The identification of a single scale $a_\star$ to account for both
      galactic kinematics and the expansion rate offset constitutes the central
      unification of this model.

      The following sections test this effective model against galactic rotation
      curve data and explore its implications for local measurements of the Hubble
      constant.

  \section{Galaxy Rotation Curves}
    \label{sec:rotation_curves}

    We now test the effective model introduced in
    Section~\ref{sec:effective_model} against observed galaxy rotation curves.
    The goal of this section is not to achieve maximal fitting flexibility, but to
    assess whether a single effective saturation scale can reproduce the main
    kinematic features of disk galaxies using baryonic matter alone.

    A useful consistency check is that the acceleration scale governing the
    transition can be expressed in terms of late-time cosmological kinematics.
    For $H_0$ in the range inferred observationally, the combination
    $a_\star \sim c H_0 / 2\pi$ is of order $10^{-10}\,\mathrm{m\,s^{-2}}$,
    comparable to the characteristic acceleration scale empirically identified in
    galaxy dynamics~\cite{McGaugh2016RAR}.
    This motivates treating the same $a_\star$ as a single scale linking galactic
    outskirts and low-density cosmic environments, without modifying early-time
    cosmology.

    \subsection{Data and Methodology}
      \label{subsec:data_method}

      We use the SPARC database, which provides high-quality rotation curves together
      with homogeneous photometry and baryonic mass models for disk galaxies spanning
      a wide range of morphologies, surface brightnesses, and dynamical
      regimes~\cite{Lelli2016SPARC}.

      The numerical pipeline used to generate the rotation-curve figures and diagnostics is available as open-source
      software~\cite{CosmochronySimulation}.

      For each galaxy, the baryonic mass distribution is constructed from the
      observed stellar and gas components.
      Stellar mass-to-light ratios are fixed following the standard SPARC
      prescriptions.
      No additional dark matter component is introduced in this effective
      description.

      The effective acceleration $g_{\mathrm{eff}}(r)$ is obtained by solving
      Eq.~\eqref{eq:effective_acceleration} using the Newtonian acceleration
      $g_{\mathrm{N}}(r)$ computed from the baryonic mass distribution.
      The circular velocity then follows from
      \begin{equation}
        v_{\mathrm{eff}}(r) = \sqrt{r\, g_{\mathrm{eff}}(r)} .
      \end{equation}

      The saturation scale $a_\star$ is treated as a universal parameter.
      Its value is fixed globally by minimizing the total residuals across the
      sample, rather than optimized on a galaxy-by-galaxy basis.

    \subsection{Representative Fits}
      \label{subsec:fits}

      Figure references are deferred to Appendix~D, where the full set of numerical
      fits is presented.
      Here we summarize the qualitative behavior observed across representative
      galaxies.

      High-surface-density galaxies are well described by Newtonian dynamics in
      their inner regions.
      Deviations from Newtonian predictions appear only at large radii, where the
      baryonic acceleration drops below the saturation scale.
      The resulting rotation curves flatten smoothly, without requiring extended
      mass halos, in agreement with observed trends across high-surface-brightness
      disks~\cite{Lelli2016SPARC}.

      Low-surface-brightness galaxies probe the saturated regime over most of their
      radial extent.
      In these systems, the effective acceleration departs from Newtonian behavior
      already at small radii, naturally producing slowly rising and asymptotically
      flat rotation curves, a feature commonly observed in diffuse systems~\cite{McGaugh2016RAR}.

      Across the sample, the model reproduces the observed diversity of rotation
      curve shapes.
      This diversity arises solely from differences in the baryonic mass
      distribution and surface density, not from variations in the effective
      parameters.

      \begin{figure}[t]
        \centering
        \includegraphics[width=\linewidth]{galaxy_rotcurves_3panel}
        \caption[Representative SPARC rotation curves]{
          Representative galaxy rotation curves from the SPARC sample.
          Observed circular velocities (points) are compared with the effective model
          prediction (solid lines) computed from the baryonic mass distribution alone.
          Left: NGC~3198 (nearly flat).
          Centre: NGC~2403 (rising).
          Right: NGC~5055 (mildly declining).
          The same universal saturation scale $a_\star$ is used for all systems.
          The only galaxy-dependent parameter is the stellar mass-to-light ratio
          $\Upsilon_\star$, fixed following standard SPARC prescriptions.
          The code used to generate these curves is publicly available
          \protect\citep{CosmochronySimulation}.
        }
        \label{fig:rotation-curves}
      \end{figure}

      Representative examples illustrating these behaviors are shown in
      Fig.~\ref{fig:rotation-curves}.

    \subsection{Residuals and Scaling Relations}
      \label{subsec:residuals}

      We quantify the quality of the fits using radial velocity residuals
      \begin{equation}
        \Delta v(r) = v_{\mathrm{obs}}(r) - v_{\mathrm{eff}}(r) .
      \end{equation}

      The residuals exhibit no systematic radial trends across the sample.
      In particular, no characteristic scale or radius-dependent bias is observed.

      The effective model naturally reproduces the empirical correlation between
      rotation curve shape and baryonic surface density.
      Galaxies with higher central surface density remain in the Newtonian regime
      over a larger radial range, while diffuse systems enter the saturated regime
      earlier.

      The transition radius at which rotation curves flatten corresponds closely to
      the regime where the effective surface flux density approaches the saturation
      limit.
      Beyond this point, further decreases in baryonic density no longer translate
      into stronger dynamical suppression, leading to asymptotically flat rotation
      velocities.

      This behavior is closely related to the observed radial acceleration relation.
      In the present framework, this relation is not imposed but emerges directly
      from the effective saturation of the gravitational response at low
      acceleration~\cite{McGaugh2016RAR}.

    \subsection{Universality of the Saturation Scale}
      \label{subsec:universality}

      A key result of the analysis is the stability of the saturation scale
      $a_\star$ across the sample.
      Allowing moderate variations of $a_\star$ does not significantly improve the
      quality of the fits and tends to introduce degeneracies with baryonic mass
      normalization.

      This supports the interpretation of $a_\star$ as a universal effective scale
      rather than a phenomenological fitting parameter.
      All galaxy-to-galaxy variability is accounted for by the observed baryonic
      structure.

      The absence of halo-by-halo tuning distinguishes this approach from standard
      dark matter halo modeling, which typically requires galaxy-dependent halo
      profiles and parameters.

      The universality of the saturation scale follows from the existence of a
      fundamental bound on the effective flux that can be supported by the geometric
      substrate.
      Unlike dark matter halo models, the present framework involves a single
      acceleration scale set by an intrinsic saturation of the underlying geometric
      response.

    \subsection{Summary}
      \label{subsec:rc_summary}

      The effective saturation model reproduces the main phenomenological features
      of galaxy rotation curves across a wide range of systems using baryonic matter
      alone.
      The same universal saturation scale governs both high- and
      low-surface-density galaxies, with smooth transitions between Newtonian and
      saturated regimes.

      These results motivate extending the analysis to cosmological low-density
      environments.
      In the next section, we show that the same effective mechanism leads to
      environment-dependent signatures in the inferred expansion rate, providing a
      natural connection to the Hubble tension.

  \section{Cosmic Voids and the Hubble Tension}
    \label{sec:voids_hubble}

    In this section we explore the cosmological implications of the effective
    saturation mechanism introduced above.
    We focus on late-time, low-density environments and show how the same effective
    dynamics responsible for flat galaxy rotation curves naturally leads to
    environment-dependent signatures in the locally inferred expansion rate.

    The analysis is restricted to phenomenology at low redshift.
    Early-universe physics and global cosmological evolution are assumed to remain
    unchanged.

    \subsection{Local Expansion in Low-Density Environments}
      \label{subsec:local_expansion}

      Measurements of the Hubble constant rely on local kinematic observables,
      including distance ladders, standard candles, and redshift--distance relations.
      These measurements implicitly assume that the local expansion rate is
      representative of the global cosmological background~\cite{Riess2022SH0ES,Freedman2021TRGB}.

      However, large-scale structure surveys show that the late-time Universe is
      highly inhomogeneous.
      A significant fraction of the cosmic volume is occupied by voids, characterized
      by matter densities well below the cosmic mean~\cite{Keenan2013KBC}.

      In the effective framework introduced in
      Section~\ref{sec:effective_model}, low-density environments probe the saturated
      regime of the gravitational response.
      As a result, kinematic relations inferred from such regions may deviate from the
      global average, even in the absence of any modification to early-time
      cosmology.

      We therefore distinguish between a global expansion rate
      $H_{\mathrm{global}}$, defined by the homogeneous background constrained by
      early-Universe observables~\cite{Planck2020}, and an effective local expansion
      rate $H_{\mathrm{loc}}$, inferred from observations within a given environment.

    \subsection{Effective Expansion Rate in Voids}
      \label{subsec:void_effect}

      Cosmic voids represent the deepest and most extended low-density regions in the
      late Universe.
      Within voids, baryonic and total matter densities are suppressed, and typical
      gravitational accelerations fall below the saturation scale $a_\star$ over large
      volumes~\cite{Keenan2013KBC}.

      In this regime, the effective dynamics modifies the relation between local
      kinematics and the underlying background expansion.
      To leading order, this effect can be captured by an environment-dependent
      renormalization of the inferred expansion rate,
      \begin{equation}
        H_{\mathrm{loc}} = H_{\mathrm{global}} \left( 1 + \delta_{\mathrm{eff}} \right) ,
        \label{eq:hloc}
      \end{equation}
      where $\delta_{\mathrm{eff}}$ is a positive correction that increases with the
      degree of saturation.

      The correction $\delta_{\mathrm{eff}}$ depends on the depth and size of the void,
      as well as on the characteristic acceleration scale $a_\star$.
      Shallower or denser regions remain close to the Newtonian regime and yield
      $\delta_{\mathrm{eff}} \approx 0$.

      Importantly, this mechanism does not require any additional energy component or
      modification of the background Friedmann equations.
      The global expansion history remains unchanged.
      Only the local inference of $H_0$ is affected.

      Within this interpretation, the enhanced local expansion inferred in cosmic
      voids reflects a reduced geometric constraint rather than an additional energy
      component.
      In regions where the density of relational nodes is low, the effective
      projection onto a geometric description operates closer to its saturation
      limit.
      This induces a local dilation of metric relations, which we refer to as a
      \emph{geometric relaxation offset}, modifying the locally inferred expansion
      rate without altering the global background evolution.

    \subsection{Connection to the Hubble Tension}
      \label{subsec:hubble_tension}

      The Hubble tension arises from the discrepancy between early-time inferences of
      the Hubble constant and late-time local measurements~\cite{Riess2022SH0ES,Planck2020}.
      Comprehensive reviews indicate that no single explanation has yet achieved
      consensus~\cite{Verde2019Review}.

      Within the present framework, this discrepancy reflects the environmental
      sensitivity of local kinematic probes.
      Early-time measurements, anchored in the homogeneous and high-density early
      Universe, recover $H_{\mathrm{global}}$.
      By contrast, late-time distance ladder measurements are performed within a
      structured Universe, often sampling void-dominated regions either directly or
      indirectly.

      If local measurements preferentially probe saturated low-density environments,
      Eq.~\eqref{eq:hloc} predicts a systematically enhanced value of the inferred
      Hubble constant,
      \begin{equation}
        H_{\mathrm{loc}} > H_{\mathrm{global}} .
      \end{equation}

      The magnitude of the offset depends on the void environment surrounding the
      observers and sources.
      This naturally explains why the tension is most prominent in local measurements
      and why it does not manifest in early-time observables.

      The extent to which local underdensities alone can account for the full tension
      remains debated in the literature~\cite{Shanks2019Void,Kennworthy2019VoidCritique}.
      The present framework provides a distinct, testable mechanism in which the
      effect arises from an effective saturation of geometric response rather than
      from a modification of the background expansion.

      Because the dynamics derives from an effective potential
      $\Phi_{\mathrm{eff}}$, the framework admits a conserved effective energy and a
      corresponding modified virial relation for stationary systems.
      This provides a consistency check that equilibrium disk galaxies remain
      dynamically stable in the transition regime.

    \subsection{Observable Signatures}
      \label{subsec:void_signatures}

      The effective saturation framework leads to several testable observational
      signatures.

      First, the inferred Hubble constant should correlate with the large-scale
      environment.
      Measurements performed in or near deep voids are expected to yield higher
      values of $H_0$ than those performed in denser regions.

      Second, the effect should exhibit redshift dependence.
      At sufficiently large redshift, line-of-sight averaging over multiple
      environments suppresses the environmental bias, and the inferred expansion rate
      should converge toward $H_{\mathrm{global}}$.

      Third, the model predicts distinct signatures in void-related observables,
      including redshift drift and weak lensing~\cite{Cai2017VoidLensing}.
      In particular, the effective expansion rate inside voids modifies the expected
      temporal evolution of redshift for sources embedded in low-density regions.

      These signatures provide independent tests of the framework and allow it to be
      falsified using current and forthcoming data.

    \subsection{Predicted Amplitude and Redshift Dependence}
      \label{subsec:hubble_predictions}

      A key feature of the present framework is that the magnitude of the locally inferred
      deviation from the global expansion rate is not an arbitrary phenomenological choice.
      Once the saturation scale $a_\star$ is fixed by galactic dynamics, the expected
      environmental offset in the inferred Hubble constant is constrained to remain at
      the percent level.

      For representative underdensities characteristic of large cosmic voids on scales of
      a few hundred megaparsecs, the framework predicts a fractional enhancement
      \begin{equation}
        \frac{\Delta H}{H}
        \;\equiv\;
        \frac{H_{\mathrm{loc}} - H_{\mathrm{global}}}{H_{\mathrm{global}}}
        \;\sim\;
        \text{a few percent} ,
      \end{equation}
      comparable in magnitude to the observed discrepancy between local distance-ladder
      measurements and early-Universe inferences.

      The effect is predicted to be redshift dependent.
      At higher redshift, the large-scale structure is less developed and line-of-sight
      averaging over multiple environments becomes increasingly efficient.
      As a result, the locally inferred expansion rate is expected to converge toward
      $H_{\mathrm{global}}$ with increasing redshift, providing a clear observational
      discriminant.

      Deviations well above the percent level or the complete absence of any measurable
      offset would directly challenge the proposed mechanism.

    \subsection{Interpretation of the Hubble Tension as an Environmental Effect}
      \label{subsec:interpretation-of-the-hubble-tension-as-an-environmental-effect}

      The framework developed here suggests that the Hubble tension does not reflect a change in the global expansion
      history,
      but rather a limitation of homogeneous kinematic inference when applied to a strongly inhomogeneous late-time
      Universe.
      Early-universe determinations of $H_0$, anchored in a nearly homogeneous background, recover the global expansion
      rate.
      By contrast, late-time local measurements probe structured environments in which the effective dynamical
      response operates in a low-density, saturated regime.

      In this context, the discrepancy between early- and late-time determinations of $H_0$ arises from a mismatch between
      global averaging and local inference.
      Measurements performed in void-dominated or underdense regions sample an effective response that no longer
      scales linearly with decreasing density, leading to a systematically enhanced locally inferred expansion rate.

      From this perspective, the Hubble tension may be viewed not as an anomaly requiring additional dark components
      or early-time modifications, but as a diagnostic of gravitational and kinematic inference in diffuse environments.
      This interpretation naturally predicts weak but systematic correlations between inferred values of $H_0$
      and large-scale environmental indicators such as void depth or filament membership.

    \subsection{Summary}
      \label{subsec:void_summary}

      In this section we have shown that the same effective saturation mechanism that
      accounts for galaxy rotation curves also leads to environment-dependent
      modifications of the locally inferred expansion rate.

      Cosmic voids emerge as natural probes of this effect and provide a structural
      connection between galactic dynamics and the Hubble tension.
      The framework preserves the global cosmological expansion while predicting
      observable deviations in local measurements.

      In the next section, we examine the robustness of these results, discuss
      degeneracies with standard cosmological effects, and outline the limits of the
      effective description.

  \section{Robustness, Degeneracies, and Limits}
    \label{sec:robustness}

    In this section we examine the robustness of the effective saturation framework,
    its degeneracies with standard astrophysical and cosmological effects, and the
    limits of its applicability.
    This assessment is essential for evaluating whether the proposed mechanism
    represents a genuine explanatory alternative or merely a reparameterization of
    existing models.

    \subsection{Baryonic Uncertainties}
      \label{subsec:baryonic_uncertainties}

      The predictions of the effective model depend explicitly on the baryonic mass
      distribution.
      Uncertainties in stellar mass-to-light ratios, gas content, and distance
      estimates therefore propagate into the inferred rotation curves.
      Such uncertainties are well documented in rotation curve analyses and are a
      generic limitation of baryon-based dynamical modeling~\cite{Lelli2016SPARC}.

      We have tested the sensitivity of the fits to reasonable variations in stellar
      mass normalization.
      While moderate shifts in mass-to-light ratio affect the detailed shape of the
      inner rotation curves, they do not eliminate the need for a low-acceleration
      modification at large radii.
      In particular, the emergence of asymptotically flat rotation curves remains
      robust across the allowed baryonic parameter range, consistent with previous
      empirical findings~\cite{McGaugh2016RAR}.

      At cosmological scales, baryonic uncertainties primarily affect the detailed
      mapping between large-scale structure and local measurements.
      They do not remove the qualitative distinction between dense environments,
      which remain close to the Newtonian regime, and voids, which probe the
      saturated regime most strongly~\cite{Keenan2013KBC}.

    \subsection{Relation to MOND-like Phenomenology}
      \label{subsec:mond_relation}

      The phenomenology described in this work shares qualitative similarities with
      acceleration-based frameworks such as Modified Newtonian Dynamics (MOND),
      originally proposed to account for flat galaxy rotation curves without
      invoking dark matter~\cite{Milgrom1983,Famaey2012Review}.

      In particular, both approaches introduce a characteristic acceleration scale
      below which departures from Newtonian expectations become significant.
      However, the present framework differs conceptually and structurally from
      MOND-like modifications of the equations of motion.

      Here, the low-acceleration behavior is not postulated as a fundamental
      modification of gravity, nor as a change in inertia.
      Instead, it arises from an effective saturation of geometric response in
      low-density environments, reflecting a limitation of the effective
      description rather than a modification of the underlying dynamical laws.

      This distinction becomes especially relevant in cosmological applications.
      Whereas MOND-based approaches face well-known challenges in consistently
      addressing large-scale structure and cosmological observations, the effective
      saturation framework preserves the standard background expansion and modifies
      only the local inference of kinematic quantities in diffuse environments.

    \subsection{Degeneracies with Standard Astrophysical Effects}
      \label{subsec:degeneracies}

      Several standard mechanisms have been proposed to address galaxy rotation
      curves and the Hubble tension within the $\Lambda$CDM framework.
      It is therefore important to assess potential degeneracies.

      At galactic scales, feedback processes and baryon--halo coupling can reproduce
      some features of flattened rotation curves within dark matter halo models,
      particularly in low-mass systems~\cite{DiCintio2014Feedback}.
      However, such mechanisms typically require galaxy-dependent tuning and do not
      naturally reproduce the observed correlation between surface density and
      dynamical behavior across the full range of disk galaxies~\cite{McGaugh2016RAR}.

      In contrast, the effective saturation model introduces no halo degrees of
      freedom and relies on a single universal scale.
      This difference leads to distinct residual patterns and scaling relations,
      which can be used to discriminate between models in detailed rotation curve
      analyses.

      At cosmological scales, local inhomogeneities, peculiar velocities, and sample
      variance are known to affect Hubble constant measurements~\cite{Freedman2021TRGB}.
      While these effects contribute to scatter, multiple studies indicate that
      their magnitude alone is generally insufficient to account for the full
      observed tension~\cite{Verde2019Review}.
      The effective saturation mechanism instead predicts a systematic
      environment-dependent bias, rather than a purely stochastic one, providing a
      clear observational discriminant.

    \subsection{Environmental Selection Effects}
      \label{subsec:selection_effects}

      Local measurements of the Hubble constant are not uniformly distributed across
      cosmic environments.
      Distance ladder calibrators and standard candles are preferentially observed
      in specific regions of the large-scale structure~\cite{Riess2022SH0ES}.

      This selection bias plays a central role in the present framework.
      If the observational strategy favors void-dominated regions, the inferred
      expansion rate will be systematically enhanced relative to the global value.

      Importantly, this is not an ad hoc assumption but a testable prediction.
      Future analyses correlating $H_0$ measurements with void catalogs and density
      reconstructions can directly assess the magnitude of this effect, as already
      explored in the context of local underdensity scenarios~\cite{Shanks2019Void,Kennworthy2019VoidCritique}.

    \subsection{Range of Validity}
      \label{subsec:validity}

      The effective saturation framework is not intended to apply universally at all
      scales and epochs.
      Its domain of validity is restricted to late-time, low-density environments.

      In high-density regimes, including the early Universe, galaxy interiors, and
      the Solar System, the model reduces to standard Newtonian and relativistic
      behavior by construction.
      No deviations from established physics are expected or required in these
      contexts, consistent with existing precision tests of gravity~\cite{Will2014Tests}.

      At sufficiently large redshift, line-of-sight averaging over multiple
      environments suppresses the effective saturation signature.
      The model therefore predicts a convergence toward standard cosmological
      behavior at high redshift, in agreement with current observations~\cite{Planck2020}.

    \subsection{Falsifiability}
      \label{subsec:falsifiability}

      The effective saturation framework makes several falsifiable predictions.

      If future measurements show no correlation between locally inferred values of
      $H_0$ and the surrounding large-scale environment, the proposed mechanism is
      disfavored.
      Similarly, if galaxy rotation curves in extremely diffuse systems deviate
      systematically from the predicted saturation behavior, the model would
      require revision or rejection.

      Conversely, confirmation of environment-dependent expansion signatures or
      consistent saturation behavior across diverse galactic systems would strongly
      support the framework.

    \subsection{Summary}
      \label{subsec:robustness_summary}

      The effective saturation model is robust against reasonable baryonic
      uncertainties and cannot be trivially absorbed into existing astrophysical or
      cosmological effects.
      Its predictive power arises from the use of a single universal scale and from
      its explicit environmental dependence.

      The framework is limited in scope but sharply testable.
      Its validity hinges on future observational probes of low-density
      environments, making it a falsifiable proposal rather than a flexible
      phenomenological fit.

  \section{Predictions and Observational Tests}
    \label{sec:predictions}

    The effective saturation framework introduced in this work leads to a set of
    distinct and testable predictions.
    These predictions arise directly from the environmental dependence of the
    effective dynamics and do not rely on additional assumptions or parameter
    tuning.

    \subsection{Environment-Dependent Hubble Measurements}
      \label{subsec:h0_environment}

      A primary prediction of the model is a correlation between the locally inferred
      Hubble constant and the surrounding large-scale environment.

      Observers and standard candles located in or near deep cosmic voids are expected
      to yield systematically higher values of $H_0$ than those located in denser
      regions.
      Conversely, measurements anchored in overdense environments should converge
      more closely toward the global expansion rate $H_{\mathrm{global}}$.

      This prediction can be tested by cross-correlating existing and forthcoming
      $H_0$ measurements with void catalogs and density reconstructions derived from
      large-scale galaxy surveys, as already explored in the context of local
      underdensity scenarios~\cite{Keenan2013KBC,Shanks2019Void,Kennworthy2019VoidCritique}.
      A null result would strongly disfavor the effective saturation mechanism.

    \subsection{Redshift Dependence of the Hubble Offset}
      \label{subsec:redshift_dependence}

      The environmental bias predicted by the model is inherently scale dependent.
      At sufficiently low redshift, local measurements are sensitive to individual
      large-scale structures and probe the saturated regime.

      At increasing redshift, line-of-sight averaging over multiple environments
      progressively suppresses the bias.
      As a result, the inferred expansion rate is predicted to converge smoothly
      toward $H_{\mathrm{global}}$.

      This behavior provides a clear observational signature.
      Measurements of $H_0$ performed over different redshift ranges should exhibit a
      systematic trend, with the largest deviations appearing at the lowest
      redshifts, consistent with expectations from environmental averaging
      arguments~\cite{Verde2019Review}.

      An important quantitative consequence of the framework is that the expected
      deviation remains at the percent level and peaks at intermediate redshift.
      Once the saturation scale is fixed by galactic dynamics, the model predicts
      that the fractional offset in the inferred expansion rate should be of order
      a few percent at $z \sim 1$, while remaining negligible at both very low and
      sufficiently high redshift.

      Such a deviation is within the sensitivity range of forthcoming large-scale
      surveys, including DESI and \textit{Euclid}, and provides a clear target for
      falsification.
      Deviations significantly larger or smaller than this level would directly
      challenge the proposed mechanism.

    \subsection{Void-Specific Redshift Drift}
      \label{subsec:redshift_drift}

      The effective saturation mechanism also affects the temporal evolution of
      redshift for sources embedded in low-density environments.

      In the standard cosmological framework, the redshift drift is determined solely
      by the global expansion history.
      In the present model, sources located in deep voids experience a modified
      effective expansion rate, leading to a small but systematic deviation in the
      expected redshift drift signal.

      Future high-precision spectroscopic experiments may be able to detect this
      environment-dependent drift by comparing sources located in voids and in
      denser regions, complementing existing proposals for redshift-drift
      measurements~\cite{Cai2017VoidLensing}.
      The effect is predicted to be absent at high redshift, where environmental
      averaging dominates.

    \subsection{Weak Lensing Signatures in Voids}
      \label{subsec:void_lensing}

      Because the effective dynamics modifies the gravitational response in
      low-density regions, it also affects weak gravitational lensing by cosmic
      voids.

      The model predicts subtle deviations in void lensing profiles compared to
      standard expectations.
      In particular, the effective saturation leads to a reduced lensing efficiency
      relative to what would be inferred from baryonic matter alone, without invoking
      additional dark components.

      Void lensing measurements therefore provide an independent probe of the
      framework, complementary to kinematic tests, and have already been identified
      as sensitive diagnostics of gravitational dynamics in underdense
      regions~\cite{Cai2017VoidLensing}.

    \subsection{Predictions for Ultra-Diffuse Galaxies}
      \label{subsec:udg}

      Ultra-diffuse galaxies represent extreme low-surface-density systems and probe
      the saturated regime across most of their extent.
      Since their systematic identification in deep imaging surveys, such systems
      have attracted considerable interest as tests of galaxy dynamics in diffuse
      environments~\cite{vanDokkum2016UDG}.

      The effective saturation model predicts that ultra-diffuse galaxies should
      exhibit slowly rising rotation curves that approach a common asymptotic
      behavior, largely independent of detailed baryonic structure.
      Significant deviations from this pattern would challenge the universality of
      the saturation scale.

      Observational studies have revealed a wide diversity of inferred dynamical mass
      content among ultra-diffuse galaxies, ranging from apparently dark-matter-rich
      systems to galaxies consistent with little or no dark matter~\cite{vanDokkum2019DF2,ManceraPina2019UDG}.
      Within the present framework, this diversity arises naturally from differences
      in how deeply individual systems probe the saturated regime.

      In ultra-diffuse galaxies, the large apparent mass discrepancy inferred from
      kinematics should therefore not be interpreted as evidence for an extended
      distribution of unseen matter.
      Rather, it reflects a geometric threshold effect: the baryonic configuration
      probes the deeply saturated regime, where further reductions in density no
      longer increase the effective dynamical response.

      As a result, the inferred dynamical mass reflects the onset of saturation
      rather than the presence of an additional gravitating component.
      The ``missing mass'' is thus not a material distribution, but an emergent
      consequence of limited geometric resolution in low-density systems.

      This interpretation leads to a falsifiable observational prediction.
      If the apparent mass discrepancy in ultra-diffuse galaxies originates from a
      geometric saturation effect rather than from a physical mass component, then
      independent mass tracers should not reveal correspondingly extended matter
      distributions.

      In particular, weak lensing signals, satellite dynamics, and stellar velocity
      dispersion profiles are expected to systematically underpredict the dynamical
      mass inferred from rotation or dispersion measurements in the deepest
      saturation regime.
      A positive detection of massive, spatially extended halos in ultra-diffuse
      galaxies would therefore directly falsify the present framework.

      Because ultra-diffuse galaxies probe the deepest saturation regime, they
      provide a particularly sensitive test of the universality of the saturation
      scale $a_\star$.

      Ultra-diffuse galaxies thus constitute a clean observational laboratory for
      distinguishing between material and geometric explanations of apparent mass
      discrepancies in the low-density regime.

    \subsection{Summary}
      \label{subsec:predictions_summary}

      The effective saturation framework yields a coherent set of predictions across
      galactic and cosmological scales.
      All predictions are driven by environmental dependence and involve no
      additional free parameters beyond the saturation scale already fixed by
      rotation curve data.

      The framework is therefore falsifiable with current or near-future
      observational capabilities.
      Confirmation or rejection of these signatures will decisively determine
      whether the proposed mechanism captures a genuine aspect of low-density
      gravitational phenomenology.

  \section{Conclusion}
    \label{sec:conclusion}

    We have investigated whether two persistent observational anomalies---flat
    galaxy rotation curves and the Hubble constant tension---may originate from a
    common effective mechanism operating in low-density environments.

    Using a minimal phenomenological framework characterized by a single saturation
    scale, we have shown that baryonic matter alone can account for the main features
    of observed galaxy rotation curves across a wide range of systems, without
    halo-by-halo tuning or the introduction of dark matter particles in this
    effective description.
    This conclusion is supported by explicit numerical comparisons with
    representative rotation curves drawn from the SPARC sample.
    The same mechanism naturally extends to cosmological low-density regions, where
    it predicts environment-dependent deviations in the locally inferred expansion
    rate.

    Cosmic voids emerge as key laboratories for testing this framework.
    By probing the deeply saturated regime, they provide a structural explanation
    for why late-time local measurements of the Hubble constant may systematically
    exceed the global value inferred from early-universe observables, while leaving
    the background cosmological evolution unchanged.

    The model is intentionally limited in scope.
    It does not modify early-time physics, does not alter the global expansion
    history, and does not claim universal validity across all regimes.
    Its strength lies instead in its parsimony and falsifiability.
    All predictions follow from a single effective scale already constrained by
    galactic dynamics.

    We have identified several observational tests capable of confirming or
    excluding this framework, including correlations between $H_0$ measurements and
    large-scale environment, redshift-dependent suppression of the local offset,
    void-specific redshift drift, weak lensing signatures, and the dynamics of
    ultra-diffuse galaxies.

    Taken together, these results suggest that galaxy rotation curves and the
    Hubble tension may reflect a shared low-density phenomenology rather than
    independent failures of the standard cosmological model.
    Whether this effective description captures a genuine aspect of gravitational
    physics or represents an intermediate phenomenological layer will ultimately be
    decided by observational tests in the coming years.

  \section*{Data Availability}

    The galaxy rotation-curve data used in this work are drawn from the publicly available
    SPARC database~\cite{Lelli2016SPARC}.

    The numerical code used to generate the galaxy rotation-curve figures and associated
    diagnostics is openly available and archived with a persistent identifier.
    The version used in this work is available via Zenodo at
    \url{https://doi.org/10.5281/zenodo.18462369}~\cite{CosmochronySimulation}.

  \section*{Acknowledgements}

    The author acknowledges the use of large language models as an editorial and analytical assistant during manuscript
    preparation, including help with phrasing alternatives, consistency checks, and structured rewriting.
    All scientific claims, modelling choices, numerical results, and interpretations are the sole responsibility of the
    author.

    \bibliographystyle{aa}
    \bibliography{references}

\end{document}
