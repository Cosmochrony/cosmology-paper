\section{Cosmic Voids and the Hubble Tension}
  \label{sec:voids_hubble}

  In this section we explore the cosmological implications of the effective
  saturation mechanism introduced above.
  We focus on late-time, low-density environments and show how the same effective
  dynamics responsible for flat galaxy rotation curves naturally leads to
  environment-dependent signatures in the locally inferred expansion rate.

  The analysis is restricted to phenomenology at low redshift.
  Early-universe physics and global cosmological evolution are assumed to remain
  unchanged.

  \subsection{Local Expansion in Low-Density Environments}
    \label{subsec:local_expansion}

    Measurements of the Hubble constant rely on local kinematic observables,
    including distance ladders, standard candles, and redshift--distance relations.
    These measurements implicitly assume that the local expansion rate is
    representative of the global cosmological background~\cite{Riess2022SH0ES,Freedman2021TRGB}.

    However, large-scale structure surveys show that the late-time Universe is
    highly inhomogeneous.
    A significant fraction of the cosmic volume is occupied by voids, characterized
    by matter densities well below the cosmic mean~\cite{Keenan2013KBC}.

    In the effective framework introduced in
    Section~\ref{sec:effective_model}, low-density environments probe the saturated
    regime of the gravitational response.
    As a result, kinematic relations inferred from such regions may deviate from the
    global average, even in the absence of any modification to early-time
    cosmology.

    We therefore distinguish between a global expansion rate
    $H_{\mathrm{global}}$, defined by the homogeneous background constrained by
    early-Universe observables~\cite{Planck2020}, and an effective local expansion
    rate $H_{\mathrm{loc}}$, inferred from observations within a given environment.

  \subsection{Effective Expansion Rate in Voids}
    \label{subsec:void_effect}

    Cosmic voids represent the deepest and most extended low-density regions in the
    late Universe.
    Within voids, baryonic and total matter densities are suppressed, and typical
    gravitational accelerations fall below the saturation scale $a_\star$ over large
    volumes~\cite{Keenan2013KBC}.

    In this regime, the effective dynamics modifies the relation between local
    kinematics and the underlying background expansion.
    To leading order, this effect can be captured by an environment-dependent
    renormalization of the inferred expansion rate,
    \begin{equation}
      H_{\mathrm{loc}} = H_{\mathrm{global}} \left( 1 + \delta_{\mathrm{eff}} \right) ,
      \label{eq:hloc}
    \end{equation}
    where $\delta_{\mathrm{eff}}$ is a positive correction that increases with the
    degree of saturation.

    The correction $\delta_{\mathrm{eff}}$ depends on the depth and size of the void,
    as well as on the characteristic acceleration scale $a_\star$.
    Shallower or denser regions remain close to the Newtonian regime and yield
    $\delta_{\mathrm{eff}} \approx 0$.

    Importantly, this mechanism does not require any additional energy component or
    modification of the background Friedmann equations.
    The global expansion history remains unchanged.
    Only the local inference of $H_0$ is affected.

    Within this interpretation, the enhanced local expansion inferred in cosmic
    voids reflects a reduced geometric constraint rather than an additional energy
    component.
    In regions where the density of relational nodes is low, the effective
    projection onto a geometric description operates closer to its saturation
    limit.
    This induces a local dilation of metric relations, which we refer to as a
    \emph{geometric relaxation offset}, modifying the locally inferred expansion
    rate without altering the global background evolution.

  \subsection{Connection to the Hubble Tension}
    \label{subsec:hubble_tension}

    The Hubble tension arises from the discrepancy between early-time inferences of
    the Hubble constant and late-time local measurements~\cite{Riess2022SH0ES,Planck2020}.
    Comprehensive reviews indicate that no single explanation has yet achieved
    consensus~\cite{Verde2019Review}.

    Within the present framework, this discrepancy reflects the environmental
    sensitivity of local kinematic probes.
    Early-time measurements, anchored in the homogeneous and high-density early
    Universe, recover $H_{\mathrm{global}}$.
    By contrast, late-time distance ladder measurements are performed within a
    structured Universe, often sampling void-dominated regions either directly or
    indirectly.

    If local measurements preferentially probe saturated low-density environments,
    Eq.~\eqref{eq:hloc} predicts a systematically enhanced value of the inferred
    Hubble constant,
    \begin{equation}
      H_{\mathrm{loc}} > H_{\mathrm{global}} .
    \end{equation}

    The magnitude of the offset depends on the void environment surrounding the
    observers and sources.
    This naturally explains why the tension is most prominent in local measurements
    and why it does not manifest in early-time observables.

    The extent to which local underdensities alone can account for the full tension
    remains debated in the literature~\cite{Shanks2019Void,Kennworthy2019VoidCritique}.
    The present framework provides a distinct, testable mechanism in which the
    effect arises from an effective saturation of geometric response rather than
    from a modification of the background expansion.

    Because the dynamics derives from an effective potential
    $\Phi_{\mathrm{eff}}$, the framework admits a conserved effective energy and a
    corresponding modified virial relation for stationary systems.
    This provides a consistency check that equilibrium disk galaxies remain
    dynamically stable in the transition regime.

  \subsection{Observable Signatures}
    \label{subsec:void_signatures}

    The effective saturation framework leads to several testable observational
    signatures.

    First, the inferred Hubble constant should correlate with the large-scale
    environment.
    Measurements performed in or near deep voids are expected to yield higher
    values of $H_0$ than those performed in denser regions.

    Second, the effect should exhibit redshift dependence.
    At sufficiently large redshift, line-of-sight averaging over multiple
    environments suppresses the environmental bias, and the inferred expansion rate
    should converge toward $H_{\mathrm{global}}$.

    Third, the model predicts distinct signatures in void-related observables,
    including redshift drift and weak lensing~\cite{Cai2017VoidLensing}.
    In particular, the effective expansion rate inside voids modifies the expected
    temporal evolution of redshift for sources embedded in low-density regions.

    These signatures provide independent tests of the framework and allow it to be
    falsified using current and forthcoming data.

  \subsection{Predicted Amplitude and Redshift Dependence}
    \label{subsec:hubble_predictions}

    A key feature of the present framework is that the magnitude of the locally inferred
    deviation from the global expansion rate is not an arbitrary phenomenological choice.
    Once the saturation scale $a_\star$ is fixed by galactic dynamics, the expected
    environmental offset in the inferred Hubble constant is constrained to remain at
    the percent level.

    For representative underdensities characteristic of large cosmic voids on scales of
    a few hundred megaparsecs, the framework predicts a fractional enhancement
    \begin{equation}
      \frac{\Delta H}{H}
      \;\equiv\;
      \frac{H_{\mathrm{loc}} - H_{\mathrm{global}}}{H_{\mathrm{global}}}
      \;\sim\;
      \text{a few percent} ,
    \end{equation}
    comparable in magnitude to the observed discrepancy between local distance-ladder
    measurements and early-Universe inferences.

    The effect is predicted to be redshift dependent.
    At higher redshift, the large-scale structure is less developed and line-of-sight
    averaging over multiple environments becomes increasingly efficient.
    As a result, the locally inferred expansion rate is expected to converge toward
    $H_{\mathrm{global}}$ with increasing redshift, providing a clear observational
    discriminant.

    Deviations well above the percent level or the complete absence of any measurable
    offset would directly challenge the proposed mechanism.

  \subsection{Interpretation of the Hubble Tension as an Environmental Effect}
    \label{subsec:interpretation-of-the-hubble-tension-as-an-environmental-effect}

    The framework developed here suggests that the Hubble tension does not reflect a change in the global expansion
    history,
    but rather a limitation of homogeneous kinematic inference when applied to a strongly inhomogeneous late-time
    Universe.
    Early-universe determinations of $H_0$, anchored in a nearly homogeneous background, recover the global expansion
    rate.
    By contrast, late-time local measurements probe structured environments in which the effective dynamical
    response operates in a low-density, saturated regime.

    In this context, the discrepancy between early- and late-time determinations of $H_0$ arises from a mismatch between
    global averaging and local inference.
    Measurements performed in void-dominated or underdense regions sample an effective response that no longer
    scales linearly with decreasing density, leading to a systematically enhanced locally inferred expansion rate.

    From this perspective, the Hubble tension may be viewed not as an anomaly requiring additional dark components
    or early-time modifications, but as a diagnostic of gravitational and kinematic inference in diffuse environments.
    This interpretation naturally predicts weak but systematic correlations between inferred values of $H_0$
    and large-scale environmental indicators such as void depth or filament membership.

  \subsection{Summary}
    \label{subsec:void_summary}

    In this section we have shown that the same effective saturation mechanism that
    accounts for galaxy rotation curves also leads to environment-dependent
    modifications of the locally inferred expansion rate.

    Cosmic voids emerge as natural probes of this effect and provide a structural
    connection between galactic dynamics and the Hubble tension.
    The framework preserves the global cosmological expansion while predicting
    observable deviations in local measurements.

    In the next section, we examine the robustness of these results, discuss
    degeneracies with standard cosmological effects, and outline the limits of the
    effective description.
