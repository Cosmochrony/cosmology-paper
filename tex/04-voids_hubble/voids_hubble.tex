\section{Cosmic Voids and the Hubble Tension}
  \label{sec:voids_hubble}

  In this section we explore the cosmological implications of the effective
  saturation mechanism introduced above.
  We focus on late-time, low-density environments and show how the same effective
  transition structure responsible for flat galaxy rotation curves naturally
  leads to environment-dependent signatures in the locally inferred expansion rate.

  The analysis is restricted to late-time phenomenology at low redshift.
  Early-universe physics and the global expansion history are assumed to remain
  unchanged.

  \subsection{Local Expansion in Low-Density Environments}
    \label{subsec:local_expansion}

    Measurements of the Hubble constant rely on local kinematic observables,
    including distance ladders, standard candles, and redshift--distance relations.
    These measurements implicitly assume that the locally inferred expansion rate
    is representative of the global cosmological background~\cite{Riess2022SH0ES,Freedman2021TRGB}.

    However, observations of large-scale structure show that the late-time Universe
    is strongly inhomogeneous.
    A substantial fraction of the cosmic volume is occupied by voids,
    characterized by matter densities well below the cosmic mean~\cite{Keenan2013KBC}.

    Within the effective framework introduced in
    Section~\ref{sec:effective_model},
    low-density environments probe the saturated regime of the effective response.
    In this regime, local kinematic observables no longer sample the same effective
    transition resolution as in high-density regions.
    As a result, expansion rates inferred from such environments need not coincide
    with the global average, even in the absence of any modification of early-time
    cosmology.

    We therefore distinguish between a global expansion rate
    $H_{\mathrm{global}}$,
    defined by the homogeneous background constrained by early-universe
    observables~\cite{Planck2020},
    and an effective local expansion rate $H_{\mathrm{loc}}$,
    inferred from observations within a specific late-time environment.

  \subsection{Effective Expansion Rate in Voids}
    \label{subsec:void_effect}

    Cosmic voids represent the deepest and most extended low-density regions in the
    late Universe.
    Within voids, baryonic and total matter densities are strongly suppressed,
    and typical gravitational accelerations fall below the saturation scale
    $a_\star$ over large volumes~\cite{Keenan2013KBC}.

    In this regime, the effective description operates close to the limited descriptive resolution in low-density
    environments.
    The relation between local kinematics and the underlying background expansion
    is therefore modified at the level of inference rather than dynamics.
    To leading order, this effect can be captured by an environment-dependent
    renormalization of the inferred expansion rate,
    \begin{equation}
      H_{\mathrm{loc}}
      =
      H_{\mathrm{global}}
      \left[
        1
        +
        \delta_\rho\,
        f\!\left(
             \frac{g_{\mathrm{void}}}{a_\star}
      \right)
      \right] ,
      \label{eq:hloc}
    \end{equation}
    Here
    \(
    \delta_\rho = 1 - \rho_{\mathrm{void}}/\rho_{\mathrm{global}}
    \)
    is the density contrast of the void,
    and
    \(
    g_{\mathrm{void}} = G M_{\mathrm{void}} / R_{\mathrm{void}}^2
    \)
    is the characteristic gravitational field strength within the void.
    The function \(f(x)\) is a dimensionless saturation function with the properties:
    \begin{itemize}
      \item \(f(x) \rightarrow 0\) for \(x \gg 1\) (unsaturated regime),
      \item \(0 < f(x) < f_{\mathrm{max}}\) for \(x \ll 1\) (saturated regime),
      \item \(f_{\mathrm{max}} \sim 0.1\), consistent with the observed amplitude of the Hubble tension.
    \end{itemize}
    The saturation scale \(a_\star\) is fixed independently by galaxy rotation curve data
    (Section~\ref{subsec:universality}).

    The correction $\delta_{\mathrm{eff}}$ depends on the depth and spatial extent of
    the void, as well as on the characteristic saturation scale $a_\star$.
    Denser or more weakly underdense regions remain close to the unsaturated regime
    and yield $\delta_{\mathrm{eff}} \approx 0$.

    Importantly, this mechanism does not require any additional energy component
    or modification of the background Friedmann equations.
    The global expansion history remains unchanged.
    Only the local inference of the expansion rate is affected.

    Within this interpretation, the enhanced expansion inferred in cosmic voids
    reflects a reduced effective geometric constraint rather than an additional
    energy density.
    In regions where the density of contributing relational structure is low,
    the kinematic inference based on sparse mass distributions operates closer to its
    saturation limit.
    This induces a systematic environment-dependent bias in the locally inferred
    kinematic relations.

  \subsection{Connection to the Hubble Tension}
    \label{subsec:hubble_tension}

    The Hubble tension arises from the discrepancy between early-time inferences of
    the Hubble constant and late-time local measurements~\cite{Riess2022SH0ES,Planck2020}.
    Comprehensive reviews indicate that no single explanation has yet achieved
    consensus~\cite{Verde2019Review}.

    Within the present framework, this discrepancy reflects the environmental
    sensitivity of late-time kinematic inference.
    Early-time measurements, anchored in the nearly homogeneous and high-density
    early Universe, recover $H_{\mathrm{global}}$.
    By contrast, late-time distance ladder measurements are performed in a
    structured Universe and may preferentially sample void-dominated regions.

    If local measurements probe environments operating in the saturated regime,
    Eq.~\eqref{eq:hloc} predicts a systematically enhanced inferred value of the
    Hubble constant,
    \begin{equation}
      H_{\mathrm{loc}} > H_{\mathrm{global}} .
    \end{equation}

    The magnitude of the offset depends on the surrounding large-scale environment
    of both observers and sources.
    This naturally explains why the tension manifests primarily in local
    measurements and not in early-time observables.

    The extent to which local underdensities alone can account for the full observed
    tension remains debated~\cite{Shanks2019Void,Kennworthy2019VoidCritique}.
    The present framework provides a distinct and testable mechanism in which the
    effect arises from saturation of the effective transition structure rather than
    from a modification of the background expansion.

    Because the dynamics derives from an effective potential
    $\Phi_{\mathrm{eff}}$,
    the framework admits a conserved effective energy and a corresponding modified
    virial relation for stationary systems.
    This provides a consistency check ensuring that equilibrium disk galaxies remain
    dynamically stable in the transition regime.

  \subsection{Observable Signatures}
    \label{subsec:void_signatures}

    The effective saturation framework leads to several testable observational
    signatures.

    First, the inferred Hubble constant should correlate with the large-scale
    environment.
    Measurements performed in or near deep voids are expected to yield higher
    values of $H_0$ than those performed in denser regions.

    Second, the effect should exhibit redshift dependence.
    At sufficiently large redshift, line-of-sight averaging over multiple
    environments suppresses the environmental bias,
    and the inferred expansion rate should converge toward
    $H_{\mathrm{global}}$.

    Third, the model predicts distinct signatures in void-related observables,
    including redshift drift and weak lensing~\cite{Cai2017VoidLensing}.
    In particular, the effective expansion rate inside voids modifies the expected
    temporal evolution of redshift for sources embedded in low-density regions.

    These signatures provide independent tests of the framework and allow it to be
    falsified using current and forthcoming data.

    More generally, the effect depends not only on the local environment of the
    observer, but on the integrated low-density structure sampled along the
    line of sight, so that both source location and intervening void geometry
    contribute to the inferred offset.

    We emphasize that the present framework does not predict a unique numerical
    value for the offset in a given environment, but constrains its scaling and
    sign once $a_\star$ is fixed.

  \subsection{Predicted Amplitude and Redshift Dependence}
    \label{subsec:hubble_predictions}

    A key feature of the present framework is that the magnitude of the locally
    inferred deviation from the global expansion rate is not an arbitrary
    phenomenological choice.
    Once the saturation scale $a_\star$ is fixed by galactic dynamics,
    the expected environmental offset in the inferred Hubble constant is
    constrained.

    For representative underdensities characteristic of large cosmic voids on
    scales of a few hundred megaparsecs,
    the framework generically allows for a fractional enhancement
    \begin{equation}
      \frac{\Delta H}{H}
      \;\equiv\;
      \frac{H_{\mathrm{loc}} - H_{\mathrm{global}}}
      {H_{\mathrm{global}}}
      \;\sim\;
      \text{a few percent} ,
    \end{equation}
    comparable in magnitude to the observed discrepancy between local distance
    ladder measurements and early-universe inferences.

    The effect is predicted to be redshift dependent.
    At higher redshift, large-scale structure is less developed and line-of-sight
    averaging becomes increasingly efficient.
    As a result, the locally inferred expansion rate is expected to converge toward
    $H_{\mathrm{global}}$ with increasing redshift.

    Deviations well above the percent level or the complete absence of any measurable
    offset would directly challenge the proposed mechanism.

  \subsection{Interpretation of the Hubble Tension as an Environmental Effect}
    \label{subsec:interpretation-of-the-hubble-tension-as-an-environmental-effect}

    The framework developed here suggests that the Hubble tension does not reflect a
    change in the global expansion history.
    It reflects a limitation of homogeneous kinematic inference when applied to a
    strongly inhomogeneous late-time Universe.

    Early-universe determinations of $H_0$,
    anchored in a nearly homogeneous background,
    recover the global expansion rate.
    By contrast, late-time local measurements probe structured environments in which
    the effective transition resolution operates in a low-density saturated regime.

    In this context, the discrepancy between early- and late-time determinations of
    $H_0$ arises from a mismatch between global averaging and local inference.
    Measurements performed in void-dominated regions sample an effective response
    that no longer scales linearly with decreasing density,
    leading to a systematically enhanced locally inferred expansion rate.

    From this perspective, the Hubble tension is not an anomaly requiring additional
    dark components or early-time modifications.
    It is a diagnostic of gravitational and kinematic inference in diffuse
    environments.
    This interpretation naturally predicts weak but systematic correlations between
    inferred values of $H_0$ and large-scale environmental indicators such as void
    depth or filament membership.

  \subsection{Summary}
    \label{subsec:void_summary}

    In this section we have shown that the same effective saturation mechanism that
    accounts for galaxy rotation curves also leads to environment-dependent
    modifications of the locally inferred expansion rate.

    Cosmic voids emerge as natural probes of this effect and provide a structural
    connection between galactic dynamics and the Hubble tension.
    The framework preserves the global cosmological expansion while predicting
    observable deviations in local measurements.

    In the next section, we examine the robustness of these results,
    discuss degeneracies with standard cosmological effects,
    and outline the limits of the effective description.
