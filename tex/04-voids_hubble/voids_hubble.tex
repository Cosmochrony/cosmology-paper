\section{Galaxy Rotation Curves}
  \label{sec:rotation_curves}

  We now test the effective model introduced in
  Section~\ref{sec:effective_model} against observed galaxy rotation curves.
  The goal of this section is not to achieve maximal fitting flexibility, but to
  assess whether a single effective saturation scale can reproduce the main
  kinematic features of disk galaxies using baryonic matter alone.

  A useful consistency check is that the acceleration scale governing the
  transition can be expressed in terms of late-time cosmological kinematics.
  For $H_0$ in the range inferred observationally, the combination
  $a_\star \sim c H_0 / 2\pi$ is of order $10^{-10}\,\mathrm{m\,s^{-2}}$,
  comparable to the characteristic acceleration scale empirically identified in
  galaxy dynamics~\cite{McGaugh2016RAR}.
  This motivates treating the same $a_\star$ as a single scale linking galactic
  outskirts and low-density cosmic environments, without modifying early-time
  cosmology.

  \subsection{Data and Methodology}
    \label{subsec:data_method}

    We use the SPARC database, which provides high-quality rotation curves together
    with homogeneous photometry and baryonic mass models for disk galaxies spanning
    a wide range of morphologies, surface brightnesses, and dynamical
    regimes~\cite{Lelli2016SPARC}.

    For each galaxy, the baryonic mass distribution is constructed from the
    observed stellar and gas components.
    Stellar mass-to-light ratios are fixed following the standard SPARC
    prescriptions.
    No additional dark matter component is introduced in this effective
    description.

    The effective acceleration $g_{\mathrm{eff}}(r)$ is obtained by solving
    Eq.~\eqref{eq:effective_acceleration} using the Newtonian acceleration
    $g_{\mathrm{N}}(r)$ computed from the baryonic mass distribution.
    The circular velocity then follows from
    \begin{equation}
      v_{\mathrm{eff}}(r) = \sqrt{r\, g_{\mathrm{eff}}(r)} .
    \end{equation}

    The saturation scale $a_\star$ is treated as a universal parameter.
    Its value is fixed globally by minimizing the total residuals across the
    sample, rather than optimized on a galaxy-by-galaxy basis.

  \subsection{Representative Fits}
    \label{subsec:fits}

    Figure references are deferred to Appendix~D, where the full set of numerical
    fits is presented.
    Here we summarize the qualitative behavior observed across representative
    galaxies.

    High-surface-density galaxies are well described by Newtonian dynamics in
    their inner regions.
    Deviations from Newtonian predictions appear only at large radii, where the
    baryonic acceleration drops below the saturation scale.
    The resulting rotation curves flatten smoothly, without requiring extended
    mass halos, in agreement with observed trends across high-surface-brightness
    disks~\cite{Lelli2016SPARC}.

    Low-surface-brightness galaxies probe the saturated regime over most of their
    radial extent.
    In these systems, the effective acceleration departs from Newtonian behavior
    already at small radii, naturally producing slowly rising and asymptotically
    flat rotation curves, a feature commonly observed in diffuse systems~\cite{McGaugh2016RAR}.

    Across the sample, the model reproduces the observed diversity of rotation
    curve shapes.
    This diversity arises solely from differences in the baryonic mass
    distribution and surface density, not from variations in the effective
    parameters.

  \subsection{Residuals and Scaling Relations}
    \label{subsec:residuals}

    We quantify the quality of the fits using radial velocity residuals
    \begin{equation}
      \Delta v(r) = v_{\mathrm{obs}}(r) - v_{\mathrm{eff}}(r) .
    \end{equation}

    The residuals exhibit no systematic radial trends across the sample.
    In particular, no characteristic scale or radius-dependent bias is observed.

    The effective model naturally reproduces the empirical correlation between
    rotation curve shape and baryonic surface density.
    Galaxies with higher central surface density remain in the Newtonian regime
    over a larger radial range, while diffuse systems enter the saturated regime
    earlier.

    The transition radius at which rotation curves flatten corresponds closely to
    the regime where the effective surface flux density approaches the saturation
    limit.
    Beyond this point, further decreases in baryonic density no longer translate
    into stronger dynamical suppression, leading to asymptotically flat rotation
    velocities.

    This behavior is closely related to the observed radial acceleration relation.
    In the present framework, this relation is not imposed but emerges directly
    from the effective saturation of the gravitational response at low
    acceleration~\cite{McGaugh2016RAR}.

  \subsection{Universality of the Saturation Scale}
    \label{subsec:universality}

    A key result of the analysis is the stability of the saturation scale
    $a_\star$ across the sample.
    Allowing moderate variations of $a_\star$ does not significantly improve the
    quality of the fits and tends to introduce degeneracies with baryonic mass
    normalization.

    This supports the interpretation of $a_\star$ as a universal effective scale
    rather than a phenomenological fitting parameter.
    All galaxy-to-galaxy variability is accounted for by the observed baryonic
    structure.

    The absence of halo-by-halo tuning distinguishes this approach from standard
    dark matter halo modeling, which typically requires galaxy-dependent halo
    profiles and parameters.

    The universality of the saturation scale follows from the existence of a
    fundamental bound on the effective flux that can be supported by the geometric
    substrate.
    Unlike dark matter halo models, the present framework involves a single
    acceleration scale set by an intrinsic saturation of the underlying geometric
    response.

  \subsection{Summary}
    \label{subsec:rc_summary}

    The effective saturation model reproduces the main phenomenological features
    of galaxy rotation curves across a wide range of systems using baryonic matter
    alone.
    The same universal saturation scale governs both high- and
    low-surface-density galaxies, with smooth transitions between Newtonian and
    saturated regimes.

    These results motivate extending the analysis to cosmological low-density
    environments.
    In the next section, we show that the same effective mechanism leads to
    environment-dependent signatures in the inferred expansion rate, providing a
    natural connection to the Hubble tension.
