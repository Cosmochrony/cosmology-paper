\section{Robustness, Degeneracies, and Limits}
  \label{sec:robustness}

  In this section we examine the robustness of the effective saturation framework,
  its degeneracies with standard astrophysical and cosmological effects, and the
  limits of its applicability.
  This assessment is essential for evaluating whether the proposed mechanism
  represents a genuine explanatory alternative or merely a reparameterization of
  existing models.

  \subsection{Baryonic Uncertainties}
    \label{subsec:baryonic_uncertainties}

    The predictions of the effective model depend explicitly on the baryonic mass
    distribution.
    Uncertainties in stellar mass-to-light ratios, gas content, and distance
    estimates therefore propagate into the inferred rotation curves.
    Such uncertainties are well documented in rotation curve analyses and are a
    generic limitation of baryon-based dynamical modeling~\cite{Lelli2016SPARC}.

    We have tested the sensitivity of the fits to reasonable variations in stellar
    mass normalization.
    While moderate shifts in mass-to-light ratio affect the detailed shape of the
    inner rotation curves, they do not eliminate the need for a low-acceleration
    modification at large radii.
    In particular, the emergence of asymptotically flat rotation curves remains
    robust across the allowed baryonic parameter range, consistent with previous
    empirical findings~\cite{McGaugh2016RAR}.

    At cosmological scales, baryonic uncertainties primarily affect the detailed
    mapping between large-scale structure and local measurements.
    They do not remove the qualitative distinction between dense environments,
    which remain close to the Newtonian regime, and voids, which probe the
    saturated regime most strongly~\cite{Keenan2013KBC}.

  \subsection{Relation to MOND-like Phenomenology}
    \label{subsec:mond_relation}

    The phenomenology described in this work shares qualitative similarities with
    acceleration-based frameworks such as Modified Newtonian Dynamics (MOND),
    originally proposed to account for flat galaxy rotation curves without
    invoking dark matter~\cite{Milgrom1983,Famaey2012Review}.

    In particular, both approaches introduce a characteristic acceleration scale
    below which departures from Newtonian expectations become significant.
    However, the present framework differs conceptually and structurally from
    MOND-like modifications of the equations of motion.

    Here, the low-acceleration behavior is not postulated as a fundamental
    modification of gravity, nor as a change in inertia.
    Instead, it arises from an effective saturation of geometric response in
    low-density environments, reflecting a limitation of the effective
    description rather than a modification of the underlying dynamical laws.

    This distinction becomes especially relevant in cosmological applications.
    Whereas MOND-based approaches face well-known challenges in consistently
    addressing large-scale structure and cosmological observations, the effective
    saturation framework preserves the standard background expansion and modifies
    only the local inference of kinematic quantities in diffuse environments.

  \subsection{Degeneracies with Standard Astrophysical Effects}
    \label{subsec:degeneracies}

    Several standard mechanisms have been proposed to address galaxy rotation
    curves and the Hubble tension within the $\Lambda$CDM framework.
    It is therefore important to assess potential degeneracies.

    At galactic scales, feedback processes and baryon--halo coupling can reproduce
    some features of flattened rotation curves within dark matter halo models,
    particularly in low-mass systems~\cite{DiCintio2014Feedback}.
    However, such mechanisms typically require galaxy-dependent tuning and do not
    naturally reproduce the observed correlation between surface density and
    dynamical behavior across the full range of disk galaxies~\cite{McGaugh2016RAR}.

    In contrast, the effective saturation model introduces no halo degrees of
    freedom and relies on a single universal scale.
    This difference leads to distinct residual patterns and scaling relations,
    which can be used to discriminate between models in detailed rotation curve
    analyses.

    At cosmological scales, local inhomogeneities, peculiar velocities, and sample
    variance are known to affect Hubble constant measurements~\cite{Freedman2021TRGB}.
    While these effects contribute to scatter, multiple studies indicate that
    their magnitude alone is generally insufficient to account for the full
    observed tension~\cite{Verde2019Review}.
    The effective saturation mechanism instead predicts a systematic
    environment-dependent bias, rather than a purely stochastic one, providing a
    clear observational discriminant.

  \subsection{Environmental Selection Effects}
    \label{subsec:selection_effects}

    Local measurements of the Hubble constant are not uniformly distributed across
    cosmic environments.
    Distance ladder calibrators and standard candles are preferentially observed
    in specific regions of the large-scale structure~\cite{Riess2022SH0ES}.

    This selection bias plays a central role in the present framework.
    If the observational strategy favors void-dominated regions, the inferred
    expansion rate will be systematically enhanced relative to the global value.

    Importantly, this is not an ad hoc assumption but a testable prediction.
    Future analyses correlating $H_0$ measurements with void catalogs and density
    reconstructions can directly assess the magnitude of this effect, as already
    explored in the context of local underdensity scenarios~\cite{Shanks2019Void,Kennworthy2019VoidCritique}.

  \subsection{Range of Validity}
    \label{subsec:validity}

    The effective saturation framework is not intended to apply universally at all
    scales and epochs.
    Its domain of validity is restricted to late-time, low-density environments.

    In high-density regimes, including the early Universe, galaxy interiors, and
    the Solar System, the model reduces to standard Newtonian and relativistic
    behavior by construction.
    No deviations from established physics are expected or required in these
    contexts, consistent with existing precision tests of gravity~\cite{Will2014Tests}.

    At sufficiently large redshift, line-of-sight averaging over multiple
    environments suppresses the effective saturation signature.
    The model therefore predicts a convergence toward standard cosmological
    behavior at high redshift, in agreement with current observations~\cite{Planck2020}.

  \subsection{Falsifiability}
    \label{subsec:falsifiability}

    The effective saturation framework makes several falsifiable predictions.

    If future measurements show no correlation between locally inferred values of
    $H_0$ and the surrounding large-scale environment, the proposed mechanism is
    disfavored.
    Similarly, if galaxy rotation curves in extremely diffuse systems deviate
    systematically from the predicted saturation behavior, the model would
    require revision or rejection.

    Conversely, confirmation of environment-dependent expansion signatures or
    consistent saturation behavior across diverse galactic systems would strongly
    support the framework.

  \subsection{Summary}
    \label{subsec:robustness_summary}

    The effective saturation model is robust against reasonable baryonic
    uncertainties and cannot be trivially absorbed into existing astrophysical or
    cosmological effects.
    Its predictive power arises from the use of a single universal scale and from
    its explicit environmental dependence.

    The framework is limited in scope but sharply testable.
    Its validity hinges on future observational probes of low-density
    environments, making it a falsifiable proposal rather than a flexible
    phenomenological fit.
