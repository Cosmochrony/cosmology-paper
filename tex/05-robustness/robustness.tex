\section{Robustness, Degeneracies, and Limits}
  \label{sec:robustness}

  In this section we assess the robustness of the effective saturation framework,
  its potential degeneracies with standard astrophysical and cosmological effects,
  and the limits of its domain of applicability.
  This analysis is essential to determine whether the proposed mechanism
  constitutes a genuine explanatory alternative
  or merely a reparameterization of existing descriptions.

  \subsection{Baryonic Uncertainties}
    \label{subsec:baryonic_uncertainties}

    The predictions of the effective framework depend explicitly on the observed baryonic mass
    distribution.
    Uncertainties in stellar mass-to-light ratios, gas content, and distance
    estimates therefore propagate into the inferred rotation curves.
    Such uncertainties are well documented
    and represent a generic limitation of baryon-based dynamical modeling~\cite{Lelli2016SPARC}.

    We have tested the sensitivity of the effective fits to reasonable variations in stellar
    mass normalization.
    While moderate shifts in mass-to-light ratio affect the detailed shape of the
    inner rotation curves, they do not eliminate the emergence of a low-acceleration
    saturation regime at large radii.
    In particular, the appearance of asymptotically flat rotation curves
    remains robust across the allowed baryonic parameter range, consistent with
    empirical scaling relations~\cite{McGaugh2016RAR}.

    At cosmological scales, baryonic uncertainties primarily affect the mapping
    between large-scale structure and local observational samples.
    They do not remove the qualitative distinction between dense environments,
    which probe an unsaturated effective regime, and cosmic voids, which sample the
    saturated regime most strongly~\cite{Keenan2013KBC}.

  \subsection{Relation to MOND-like Phenomenology}
    \label{subsec:mond_relation}

    The phenomenology discussed in this work shares qualitative similarities with
    acceleration-based frameworks such as Modified Newtonian Dynamics (MOND),
    originally introduced to account for flat galaxy rotation curves without
    invoking dark matter~\cite{Milgrom1983,Famaey2012Review}.

    In particular, both approaches identify a characteristic acceleration scale
    below which Newtonian expectations fail.
    However, the conceptual status of this scale differs fundamentally.

    Three key observational discriminants distinguish the present framework from MOND:
    \begin{itemize}
      \item \textbf{Environment-dependent Hubble measurements}
      : The effective framework predicts a correlation between locally inferred \(H_0\)
      and large-scale environment (Eq.~\eqref{eq:hloc}), absent in MOND where \(a_0\)
      is universal and does not couple to cosmological expansion.

      \item \textbf{Redshift dependence of the Hubble offset}: The deviation \(\Delta H/H\)
      is predicted to converge to zero at high redshift (Section~\ref{subsec:redshift_dependence}
      ), while MOND provides no mechanism for redshift-dependent modifications of \(H_0\).

      \item \textbf{Ultra-diffuse galaxy kinematics}: UDGs in voids are predicted to exhibit systematically lower
      asymptotic velocities than UDGs in clusters(Section~\ref{subsec:udg}), whereas MOND predicts a universal
      \(v_{\infty} \propto (M a_0)^{1/4}\) independent of environment.
    \end{itemize}
    These differences arise because the saturation scale \(a_\star\) in the effective framework is tied to the
    \textbf{projection efficiency}
    of the underlying geometric substrate (Section 4.3 of [Beau, 2026c]), not to a modified force law.

    In the present framework, the low-acceleration behavior is not postulated
    as a modification of gravity or inertia.
    It arises as a saturation of the effective transition resolution
    in low-density environments.
    The scale $a_\star$ characterizes a limit of descriptive refinement,
    not a new dynamical constant.

    This distinction becomes especially important in cosmological applications.
    MOND-based extensions encounter well-known difficulties
    when applied consistently to large-scale structure
    and expansion observables.
    By contrast,
    the effective saturation framework preserves
    the standard background expansion
    and modifies only the local inference of kinematic quantities
    in diffuse late-time environments.

  \subsection{Degeneracies with Standard Astrophysical Effects}
    \label{subsec:degeneracies}

    Several mechanisms within the $\Lambda$CDM framework
    have been proposed to address galaxy rotation curves
    and the Hubble tension.
    It is therefore necessary to examine possible degeneracies.

    At galactic scales, feedback processes and baryon--halo coupling can reproduce
    some features of flattened rotation curves within dark matter halo models,
    particularly in low-mass systems~\cite{DiCintio2014Feedback}.
    Such mechanisms typically require galaxy-dependent tuning and do not
    naturally reproduce the observed correlation between surface density and
    dynamical response across the full range of disk galaxies~\cite{McGaugh2016RAR}.

    In contrast, the effective saturation framework introduces no halo degrees of
    freedom and relies on a single universal scale.
    All diversity in rotation curve shapes arises from the observed baryonic distributions.
    This leads to distinct residual patterns and scaling relations,
    which can be used to discriminate between models in detailed rotation curve
    analyses.

    At cosmological scales, local inhomogeneities, peculiar velocities, and sample
    variance are known to affect Hubble constant measurements~\cite{Freedman2021TRGB}.
    While these effects contribute to statistical scatter,
    multiple studies indicate that they are generally insufficient
    to account for the full observed tension~\cite{Verde2019Review}.
    The effective saturation framework instead predicts
    a systematic environment-dependent bias, reflecting a change in effective transition resolution
    rather than a stochastic fluctuation.

    A key aspect of the present framework is that the fundamental bound $b$
    is universal.
    It fixes a constant maximal resolvability of the effective projection,
    which may be viewed as a uniform ``pixel density'' of the geometric description
    of the Universe.

    Importantly, different environments do not exploit this fixed resolution
    in the same way.
    Dense systems, such as galaxies or clusters, concentrate many effective
    relations within a limited physical volume.
    In these regimes, the available resolution is primarily consumed by internal
    structure, and the effective transition rate remains close to the Newtonian
    or background behavior.

    By contrast, cosmic voids distribute the same fixed resolution over much larger
    volumes.
    The effective density of contributing relations is therefore reduced, and the
    projection operates closer to its saturation threshold.
    This leads to an enhanced effective coupling in low-density environments,
    without any change in the underlying bound $b$ or in the global expansion
    history.

    This distinction cannot be absorbed into standard astrophysical degeneracies.
    It reflects a difference in how environments of contrasting density sample
    a fixed geometric resolution, rather than a reparameterization of existing
    dynamical models.

  \subsection{Environmental Selection Effects}
    \label{subsec:selection_effects}

    Local measurements of the Hubble constant are not uniformly distributed across
    cosmic environments.
    Distance ladder calibrators and standard candles are preferentially observed
    in specific regions of the large-scale structure~\cite{Riess2022SH0ES}.

    This selection effect plays a central role in the present framework.
    If observational strategies preferentially sample
    void-dominated or underdense regions,
    the inferred expansion rate is expected
    to be systematically enhanced relative to the global value.

    Importantly, this effect is not introduced as an ad hoc assumption.
    It constitutes a direct and testable prediction.
    Future analyses correlating inferred values of $H_0$ with void catalogs and density
    reconstructions can directly assess the magnitude of the effect, as already
    explored in the context of local underdensity scenarios~\cite{Shanks2019Void,Kennworthy2019VoidCritique}.

  \subsection{Range of Validity}
    \label{subsec:validity}

    The effective saturation framework is not intended to apply universally at all
    scales and epochs.
    Its domain of validity is restricted to late-time, low-density environments.

    In high-density regimes, including the early Universe, galaxy interiors, and
    the Solar System, the effective description reduces to standard Newtonian and relativistic
    behavior by construction.
    No deviations from established physics are expected in these
    contexts, consistent with precision tests of gravity~\cite{Will2014Tests}.

    At sufficiently large redshift, line-of-sight averaging over multiple
    environments suppresses the saturation signature.
    The framework therefore predicts convergence toward standard cosmological
    behavior at high redshift, in agreement with current observations~\cite{Planck2020}.

  \subsection{Falsifiability}
    \label{subsec:falsifiability}

    The effective saturation framework makes several falsifiable predictions.

    If future measurements show no correlation between locally inferred values of
    $H_0$ and the surrounding large-scale environment, the proposed mechanism is
    disfavored.
    Similarly, if galaxy rotation curves in extremely diffuse systems systematically deviate
    from the predicted saturation behavior, the framework would
    require revision or rejection.

    Conversely, the confirmation of environment-dependent expansion signatures or
    consistent saturation behavior across diverse galactic systems would provide strong
    support for the model.

  \subsection{Summary}
    \label{subsec:robustness_summary}

    The effective saturation framework is robust against reasonable baryonic
    uncertainties and cannot be trivially absorbed into existing astrophysical or
    cosmological effects.
    Its explanatory power arises from the use of a single universal scale and from
    its explicit environmental dependence.

    The framework is limited in scope but sharply testable.
    Its validity hinges on future observational probes of low-density
    environments, making it a falsifiable proposal rather than a flexible
    phenomenological fit.
