\section{Introduction}
  \label{sec:introduction}

  Two observational puzzles continue to motivate scrutiny of late-time
  gravitational phenomenology.

  The first is the ubiquity of flat galaxy rotation curves.
  Across a wide range of disk galaxies, circular velocities remain nearly
  constant at large radii, in apparent tension with the expectation from
  Newtonian dynamics applied to the observed baryonic mass distribution.
  Within the standard cosmological framework, this behavior is typically
  accounted for by extended dark matter halos.
  While this approach is empirically successful, it introduces substantial
  model freedom at the level of individual galaxies and leaves open the question
  of why tight empirical regularities, such as surface-density-dependent trends
  and acceleration-based scalings, emerge so prominently in the data.

  The second puzzle is the Hubble constant tension.
  Late-time local determinations of $H_0$ based on distance ladder techniques
  and standard candles remain in significant disagreement with early-time
  inferences derived from the homogeneous high-redshift Universe.
  This discrepancy has persisted across multiple datasets and analysis methods,
  prompting proposals ranging from systematic effects to extensions of the
  standard cosmological model.
  A notable feature of the tension is its strong association with late-time
  inference in a structured, inhomogeneous Universe.

  Although galaxy rotation curves and the Hubble tension are usually discussed
  separately, both phenomena probe gravitational and kinematic inference in
  low-density regimes.
  Galaxy outskirts and ultra-diffuse systems explore weak-acceleration regions,
  while cosmic voids represent the most extreme low-density environments on
  cosmological scales.
  This motivates the question addressed in this work.
  Can a single effective mechanism, active primarily in diffuse environments,
  account for both classes of anomalies within a unified phenomenological
  description?

  This phenomenological study is motivated by a broader theoretical context in
  which bounded-response regimes arise naturally in effective descriptions of
  gravitational dynamics~\cite{Beau2026c}.
  Related structural considerations, including the role of coarse-grained
  descriptions and projection effects, are discussed in companion works but are
  not required for the present analysis~\cite{Beau2026a,Beau2026b}.
  Here we deliberately restrict attention to late-time observational
  consequences and do not rely on the full underlying framework.

  In this paper we investigate a minimal effective model in which the
  gravitational response saturates below a characteristic scale.
  The model is constructed to reduce to standard Newtonian behavior in
  high-density environments and to introduce a smooth bounded modification in
  diffuse regimes.
  It is treated explicitly as an effective description of late-time
  phenomenology.
  No modification of early-universe physics is assumed, and the model is not
  presented as a fundamental theory of gravity.

  The analysis proceeds in two stages.
  First, we test the model against galaxy rotation curve data, focusing on the
  SPARC sample, and evaluate whether a single universal saturation scale can
  reproduce the observed diversity of rotation curve shapes using baryonic matter
  alone, without halo-by-halo tuning.
  Second, we explore the cosmological implications of the same mechanism in
  low-density environments, emphasizing cosmic voids as maximal probes and
  deriving testable signatures for environment-dependent expansion inference.

  A key aim of this work is falsifiability.
  Beyond fitting existing data, we identify observational discriminants that can
  confirm or exclude the framework.
  These include correlations between locally inferred values of $H_0$ and
  large-scale environment, redshift-dependent suppression of the local offset,
  void-specific redshift drift, and weak lensing signatures.

  The structure of the paper is as follows.
  Section~\ref{sec:effective_model} introduces the effective model and its minimal
  assumptions.
  Section~\ref{sec:rotation_curves} presents rotation curve tests and diagnostics.
  Section~\ref{sec:voids_hubble} discusses the implications for cosmic voids and
  the Hubble tension.
  Section~\ref{sec:robustness} examines robustness, degeneracies, and limits.
  Section~\ref{sec:predictions} summarizes predictions and observational tests.
  Section~\ref{sec:conclusion} concludes.
