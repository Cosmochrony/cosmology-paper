\section{Introduction}
  \label{sec:introduction}

  Two observational puzzles continue to motivate scrutiny of late-time
  gravitational phenomenology.

  The first is the ubiquity of flat galaxy rotation curves.
  Across a wide range of disk galaxies, circular velocities remain nearly
  constant at large radii, in apparent tension with the expectation from
  Newtonian dynamics applied to the observed baryonic mass distribution.
  This behavior has been firmly established by large and homogeneous datasets,
  most notably the SPARC compilation, which provides resolved rotation curves
  together with self-consistent baryonic mass models for disk galaxies~\cite{Lelli2016SPARC}.
  Within the standard cosmological framework, flat rotation curves are typically
  accounted for by extended dark matter halos.
  While this approach is empirically successful, it introduces substantial model
  freedom at the level of individual galaxies and leaves open the question of why
  tight empirical regularities, such as surface-density-dependent trends and
  acceleration-based scalings, emerge so prominently in the data~\cite{McGaugh2016RAR}.

  The second puzzle is the Hubble constant tension.
  Late-time local determinations of $H_0$ based on distance ladder techniques and
  standard candles remain in significant disagreement with early-time inferences
  derived from the homogeneous high-redshift Universe as constrained by cosmic
  microwave background observations~\cite{Riess2022SH0ES,Planck2020}.
  This discrepancy has persisted across multiple datasets and analysis methods,
  prompting proposals ranging from unidentified systematics to extensions of the
  standard cosmological model.
  Comprehensive reviews indicate that no consensus explanation has yet emerged~\cite{Verde2019Review}.
  A notable feature of the tension is its strong association with late-time
  inference in a structured and inhomogeneous Universe.

  Although galaxy rotation curves and the Hubble tension are usually discussed
  separately, both phenomena probe gravitational and kinematic inference in
  low-density regimes.
  Galaxy outskirts and ultra-diffuse systems explore weak-acceleration regions,
  while cosmic voids represent the most extreme low-density environments on
  cosmological scales.
  This motivates the central question addressed in this work.
  Can a single effective mechanism, active primarily in diffuse environments,
  account for both classes of anomalies within a unified phenomenological
  description?

  In this paper we investigate a minimal effective framework in which departures
  from Newtonian dynamics arise in low-density environments.
  In such regimes, the effective gravitational response does not continue to grow
  indefinitely as density decreases, but instead approaches a plateau controlled
  by an intrinsic saturation scale.

  This behavior is not introduced as an ad hoc modification of the force law.
  Rather, it reflects a limitation in the effective description applicable to
  diffuse regimes, where dynamical inference becomes progressively insensitive to
  further reductions in density.
  As a result, the effective force remains finite at large radii, producing the
  characteristic flattening observed in dynamical observables.

  This phenomenological study is motivated by a broader theoretical context in
  which bounded-response regimes arise naturally in effective descriptions of
  gravitational dynamics~\cite{Beau2026c}.
  Related structural and projection-based considerations are discussed in
  companion works but are not required for the present analysis~\cite{Beau2026a,Beau2026b}.
  Here we deliberately restrict attention to late-time observational
  consequences and do not rely on the full underlying framework.

  The model is constructed to reduce to standard Newtonian behavior in
  high-density environments and to introduce a smooth bounded modification in
  diffuse regimes.
  It is treated explicitly as an effective description of late-time
  phenomenology.
  No modification of early-universe physics is assumed, and the model is not
  presented as a fundamental theory of gravity.

  The analysis proceeds in two stages.
  First, we test the model against galaxy rotation curve data, focusing on the
  SPARC sample, and evaluate whether a single universal saturation scale can
  reproduce the observed diversity of rotation curve shapes using baryonic matter
  alone, without halo-by-halo tuning.
  Second, we explore the cosmological implications of the same mechanism in
  low-density environments, emphasizing cosmic voids as maximal probes and
  deriving testable signatures for environment-dependent expansion inference.

  A key aim of this work is falsifiability.
  Beyond fitting existing data, we identify observational discriminants that can
  confirm or exclude the framework.
  These include correlations between locally inferred values of $H_0$ and
  large-scale environment, redshift-dependent suppression of the local offset,
  void-specific redshift drift, and weak lensing signatures.

  The structure of the paper is as follows.
  Section~\ref{sec:effective_model} introduces the effective model and its minimal
  assumptions.
  Section~\ref{sec:rotation_curves} presents rotation curve tests and diagnostics.
  Section~\ref{sec:voids_hubble} discusses the implications for cosmic voids and
  the Hubble tension.
  Section~\ref{sec:robustness} examines robustness, degeneracies, and limits.
  Section~\ref{sec:predictions} summarizes predictions and observational tests.
  Section~\ref{sec:conclusion} concludes.
