\section{Predictions and Observational Tests}
  \label{sec:predictions}

  The effective saturation framework introduced in this work leads to a set of
  distinct and testable predictions.
  These predictions arise directly from the environmental dependence of the
  effective dynamics and do not rely on additional assumptions or parameter
  tuning.

  \subsection{Environment-Dependent Hubble Measurements}
    \label{subsec:h0_environment}

    A primary prediction of the model is a correlation between the locally inferred
    Hubble constant and the surrounding large-scale environment.

    Observers and standard candles located in or near deep cosmic voids are expected
    to yield systematically higher values of $H_0$ than those located in denser
    regions.
    Conversely, measurements anchored in overdense environments should converge
    more closely toward the global expansion rate $H_{\mathrm{global}}$.

    This prediction can be tested by cross-correlating existing and forthcoming
    $H_0$ measurements with void catalogs and density reconstructions derived from
    large-scale galaxy surveys.
    A null result would strongly disfavor the effective saturation mechanism.

  \subsection{Redshift Dependence of the Hubble Offset}
    \label{subsec:redshift_dependence}

    The environmental bias predicted by the model is inherently scale dependent.
    At sufficiently low redshift, local measurements are sensitive to individual
    large-scale structures and probe the saturated regime.

    At increasing redshift, line-of-sight averaging over multiple environments
    progressively suppresses the bias.
    As a result, the inferred expansion rate is predicted to converge smoothly
    toward $H_{\mathrm{global}}$.

    This behavior provides a clear observational signature.
    Measurements of $H_0$ performed over different redshift ranges should exhibit a
    systematic trend, with the largest deviations appearing at the lowest redshifts.

  \subsection{Void-Specific Redshift Drift}
    \label{subsec:redshift_drift}

    The effective saturation mechanism also affects the temporal evolution of
    redshift for sources embedded in low-density environments.

    In the standard cosmological framework, the redshift drift is determined solely
    by the global expansion history.
    In the present model, sources located in deep voids experience a modified
    effective expansion rate, leading to a small but systematic deviation in the
    expected redshift drift signal.

    Future high-precision spectroscopic experiments may be able to detect this
    environment-dependent drift by comparing sources located in voids and in denser
    regions.
    The effect is predicted to be absent at high redshift, where environmental
    averaging dominates.

  \subsection{Weak Lensing Signatures in Voids}
    \label{subsec:void_lensing}

    Because the effective dynamics modifies the gravitational response in
    low-density regions, it also affects weak gravitational lensing by cosmic voids.

    The model predicts subtle deviations in void lensing profiles compared to
    standard expectations.
    In particular, the effective saturation leads to a reduced lensing efficiency
    relative to what would be inferred from baryonic matter alone, without invoking
    additional dark components.

    Void lensing measurements therefore provide an independent probe of the
    framework, complementary to kinematic tests.

  \subsection{Predictions for Ultra-Diffuse Galaxies}
    \label{subsec:udg}

    Ultra-diffuse galaxies represent extreme low-surface-density systems and probe
    the saturated regime across most of their extent.

    The effective saturation model predicts that such galaxies should exhibit
    slowly rising rotation curves that approach a common asymptotic behavior,
    largely independent of detailed baryonic structure.
    Significant deviations from this pattern would challenge the universality of
    the saturation scale.

  \subsection{Summary}
    \label{subsec:predictions_summary}

    The effective saturation framework yields a coherent set of predictions across
    galactic and cosmological scales.
    All predictions are driven by environmental dependence and involve no additional
    free parameters beyond the saturation scale already fixed by rotation curve
    data.

    The framework is therefore falsifiable with current or near-future
    observational capabilities.
    Confirmation or rejection of these signatures will decisively determine whether
    the proposed mechanism captures a genuine aspect of low-density gravitational
    phenomenology.
