\section{Predictions and Observational Tests}
  \label{sec:predictions}

  The effective saturation framework introduced in this work leads to a set of
  distinct and testable predictions.
  These predictions follow directly from the environmental dependence of the
  effective transition resolution and do not rely on additional assumptions or
  parameter tuning.

  \subsection{Environment-Dependent Hubble Measurements}
    \label{subsec:h0_environment}

    A primary prediction of the framework is a correlation between the locally inferred
    Hubble constant and the surrounding large-scale environment.

    Observers and standard candles located in or near deep cosmic voids are expected
    to yield systematically higher inferred values of $H_0$ than those located in denser
    regions.
    Conversely, measurements anchored in overdense environments should converge
    more closely toward the global expansion rate $H_{\mathrm{global}}$.

    This prediction can be tested by cross-correlating existing and forthcoming
    $H_0$ measurements with void catalogs and density reconstructions derived from
    large-scale galaxy surveys, as already explored in the context of local
    underdensity scenarios~\cite{Keenan2013KBC,Shanks2019Void,Kennworthy2019VoidCritique}.
    A statistically significant null result would strongly disfavor the effective
    saturation mechanism.

    In particular, the framework predicts a differential signal:
    measurements performed in environments of comparable redshift but contrasting
    large-scale density should exhibit systematically different inferred values of $H_0$.

  \subsection{Redshift Dependence of the Hubble Offset}
    \label{subsec:redshift_dependence}

    The environmental bias predicted by the framework is inherently scale dependent.

    At sufficiently low redshift, local measurements are sensitive to individual
    large-scale structures and directly probe the saturated regime.
    At increasing redshift, line-of-sight averaging over multiple environments
    progressively suppresses the bias.

    As a result, the inferred expansion rate is predicted to converge smoothly
    toward $H_{\mathrm{global}}$ with increasing redshift.

    This behavior provides a clear observational signature.
    Measurements of the Hubble constant performed over different redshift ranges
    should exhibit a systematic trend, with the largest deviations appearing at the
    lowest redshifts, followed by a gradual convergence at higher
    redshift~\cite{Verde2019Review}.

    An important quantitative consequence of the framework is that the expected
    deviation remains at the percent level.
    Once the saturation scale $a_\star$ is fixed by galactic dynamics, the fractional
    offset in the inferred expansion rate is predicted to peak at intermediate
    redshift and to remain negligible at both very low and sufficiently high redshift.

    The peak is expected when the typical line-of-sight length becomes comparable
    to the characteristic scale of large voids, before full environmental averaging
    sets in.

    Such deviations fall within the sensitivity range of forthcoming large-scale
    surveys, including DESI and \textit{Euclid}.
    Observed deviations significantly larger or smaller than this level would
    directly challenge the proposed mechanism.

  \subsection{Void-Specific Redshift Drift}
    \label{subsec:redshift_drift}

    The effective saturation mechanism also affects the temporal evolution of
    redshift for sources embedded in low-density environments.

    In the standard cosmological framework, the redshift drift depends solely on the
    global expansion history.
    In the present framework, sources located in deep voids experience a modified
    effective transition rate, leading to a small but systematic deviation in the
    expected redshift drift signal.

    Future high-precision spectroscopic experiments may be able to detect this
    effect by comparing sources located in voids with those in denser regions,
    complementing existing redshift-drift proposals~\cite{Cai2017VoidLensing}.
    The effect is predicted to be suppressed at high redshift, where environmental
    averaging dominates.

  \subsection{Weak Lensing Signatures in Voids}
    \label{subsec:void_lensing}

    Because the effective framework modifies the inferred gravitational response
    in low-density regions, it also impacts weak gravitational lensing by cosmic
    voids.

    The model predicts subtle but systematic deviations in void lensing profiles
    relative to standard expectations.
    In particular, the effective saturation leads to a systematically altered
    lensing efficiency compared to what would be inferred from baryonic matter
    alone, without invoking additional dark components.

    Void lensing measurements therefore provide an independent probe of the
    framework, complementary to kinematic tests, and have been identified as
    sensitive diagnostics of gravitational behavior in underdense
    regions~\cite{Cai2017VoidLensing}.

  \subsection{Predictions for Ultra-Diffuse Galaxies}
    \label{subsec:udg}

    Ultra-diffuse galaxies constitute extreme low-surface-density systems and
    probe the saturated regime across most of their spatial
    extent~\cite{vanDokkum2016UDG}.

    Quantitatively, in the saturated regime ($g_{\mathrm{N}} \ll a_\star$), the
    effective framework predicts that the asymptotic circular velocity is bounded
    by the finite response of the effective gravitational field.
    To leading order, this implies
    \begin{equation}
      v_{\infty}
      \;\sim\;
      \sqrt{R\, a_\star}\;
      \left[
        1
        -
        \mathcal{O}\!\left(\delta_\rho\right)
      \right] ,
      \label{eq:udg-velocity}
    \end{equation}
    where $R$ is the characteristic baryonic scale length of the system and
    $\delta_\rho = 1 - \rho_{\mathrm{void}}/\rho_{\mathrm{global}}$ is the local
    underdensity contrast.
    For ultra-diffuse galaxies in void environments
    ($\delta_\rho \approx 0.3$--$0.6$ and $R \sim 3$--$5$~kpc),
    this yields typical asymptotic velocities
    $v_{\infty} \lesssim 10$--$15~\mathrm{km\,s^{-1}}$.

    This prediction differs qualitatively from MOND, which enforces a universal
    relation $v_{\infty} \propto (M_{\mathrm{baryon}} a_0)^{1/4}$ independent of
    environment.
    In the present framework, the asymptotic velocity depends explicitly on the
    large-scale environment through $\delta_\rho$, leading to systematically lower
    velocities for ultra-diffuse galaxies in voids than for systems of comparable
    baryonic mass in denser regions.

    Observations reveal a wide diversity of inferred dynamical mass content among
    ultra-diffuse galaxies, ranging from apparently dark-matter-rich systems to
    galaxies consistent with little or no dark
    matter~\cite{vanDokkum2019DF2,ManceraPina2019UDG}.
    Within the present framework, this diversity arises naturally from differences
    in how deeply individual systems probe the saturated regime.

    In ultra-diffuse galaxies, the apparent mass discrepancy inferred from
    kinematics should therefore not be interpreted as evidence for an extended
    distribution of unseen matter.
    It reflects a geometric threshold effect.
    The baryonic configuration probes a regime in which further density dilution no
    longer increases the effective dynamical response.

    As a result, the inferred dynamical mass tracks the onset of saturation rather
    than the presence of an additional gravitating component.
    The apparent ``missing mass'' is thus an emergent consequence of limited
    descriptive resolution in low-density systems.

    This interpretation leads to a falsifiable prediction.
    If the inferred mass discrepancy originates from geometric saturation rather
    than from a material mass component, independent mass tracers should not reveal
    correspondingly extended matter distributions.

    In particular, weak lensing signals, satellite dynamics, and stellar velocity
    dispersion profiles are expected to systematically underpredict the dynamical
    mass inferred from kinematic measurements in the deepest saturation regime.
    A positive detection of massive, spatially extended halos in ultra-diffuse
    galaxies would directly falsify the present framework.

    Because ultra-diffuse galaxies probe the deepest saturation regime, they
    provide a particularly sensitive test of the universality of the saturation
    scale $a_\star$.

  \subsection{Summary}
    \label{subsec:predictions_summary}

    The effective saturation framework yields a coherent and tightly constrained
    set of predictions across both galactic and cosmological scales.

    All predictions arise from environmental dependence and involve no additional
    free parameters beyond the saturation scale $a_\star$ already fixed by rotation
    curve data.

    The framework is therefore falsifiable with current or near-future
    observational capabilities.
    Confirmation or rejection of these signatures will decisively determine
    whether the proposed mechanism captures a genuine aspect of low-density
    gravitational phenomenology.
